% Created 2015-02-09 lun 07:54
\documentclass[spanish,presentation]{beamer}
\usepackage[utf8]{inputenc}
\usepackage[T1]{fontenc}
\usepackage{fixltx2e}
\usepackage{graphicx}
\usepackage{longtable}
\usepackage{float}
\usepackage{wrapfig}
\usepackage{rotating}
\usepackage[normalem]{ulem}
\usepackage{amsmath}
\usepackage{textcomp}
\usepackage{marvosym}
\usepackage{wasysym}
\usepackage{amssymb}
\usepackage{capt-of}
\tolerance=1000
\usepackage{listings}
\usepackage[mono=false]{libertine}
\usepackage[scaled=0.7]{luximono}
\usepackage{lxfonts}
\usepackage[spanish, mexico, es-noshorthands, english]{babel}
% remove space between margin and lists
\usepackage{enumitem}
\setitemize{label=\usebeamerfont*{itemize item}%
\usebeamercolor[fg]{itemize item}
\usebeamertemplate{itemize item}}
\setlist{leftmargin=*,labelindent=0cm}
\setenumerate[1]{%
label=\protect\usebeamerfont{enumerate item}%
\protect\usebeamercolor[fg]{enumerate item}%
\insertenumlabel.}
\usepackage{tikz}
\usepackage{tikz-cd}
\usepackage{pgfplots}
\usetikzlibrary{babel}
\beamerdefaultoverlayspecification{<+->}
\usefonttheme{professionalfonts}
\usetheme{default}
\usecolortheme{shark}
\useinnertheme[outline,shadow]{chamfered}
\useoutertheme[glossy,nofootline]{wuerzburg}
\date{2015-01-30 7:00}
\title{Funciones derivables}
\begin{document}

\maketitle
\setbeamertemplate{navigation symbols}{}
\setbeamertemplate{items}[circle]
\languagepath{spanish}

\tableofcontents

\section{Definición}
\label{sec-1}

\begin{frame}[label=sec-1-1]{Derivada}
\begin{definition}
Sean \(A\subseteq \mathbb{C}\) y \(z_{0}\) en el interior de
\(A\). Decimos que \(f\colon A\to \mathbb{C}\) es \alert{derivable en
\(z_{0}\)} si el límite
\begin{displaymath}
\lim_{z\to z_{0}}\frac{f(z)-f(z_{0})}{z-z_{0}}
\end{displaymath}
existe. En caso de existir, al valor del límite lo denotamos por
\(f'(z_{0})\). 
\end{definition}
\end{frame}

\begin{frame}[label=sec-1-2]{Equivalencias de la definición}
\begin{theorem}
Dados \(A\subseteq \mathbb{C}\) y \(z_{0}\) en el interior de
\(A\), son equivalentes:

\begin{itemize}
\item \(f\) es derivable en \(z_{0}\).
\item El límite 
\begin{displaymath}
\lim_{h\to 0}\frac{f(z_{0}+h)-f(z_{0})}{h}
\end{displaymath}
existe.
\item Existen \(c\in \mathbb{C}\) y una función \(E\colon A\to
      \mathbb{C}\) tal que
\begin{displaymath}
f(z)=f(z_{0})+c(z-z_{0})+E(z)
\end{displaymath}
con \(\lim_{z\to z_{0}}\frac{E(z)}{z-z_{0}}=0\).
\end{itemize}
\end{theorem}
\end{frame}

\begin{frame}[label=sec-1-3]{Observaciones}
\begin{itemize}
\item Dada \(f\colon A\to \mathbb{C}\), se define \(f'\colon B\to
     \mathbb{C}\), donde \(B\subseteq A\) es el conjunto de puntos
donde \(f\) es derivable.
\item Si \(f\) es derivable en \(z_{0}\), entonces es continua en \(z_{0}\).
\end{itemize}
\end{frame}

\section{Reglas de derivación}
\label{sec-2}

\begin{frame}[label=sec-2-1]{}
\begin{theorem}
Supongamos que \(f,g\colon A\to \mathbb{C}\) son derivables en
\(z_{0}\in A\) y \(c\in \mathbb{C}\). Entonces \(cf\), \(f+g\),
\(fg\) son derivables en \(z_{0}\). También \(\frac{f}{g}\) es
derivable en \(z_{0}\) si \(g(z_{0})\ne 0\). Además:

\begin{itemize}
\item \((cf)'(z_{0})=cf'(z_{0})\),
\item \((f+g)(z_{0})=f(z_{0})+g(z_{0})\),
\item \((fg)'(z_{0})=f'(z_{0})g(z_{0})+f(z_{0})g'(z_{0})\),
\item \((\frac{f}{g})'(z_{0})=\frac{f'(z_{0})g(z_{0})-f(z_{0})g'(z_{0})}{g(z_{0})^{2}}\).
\end{itemize}
\end{theorem}

\begin{theorem}
Para todo \(z\in \mathbb{Z}\) se tiene que \(f(z)=z^{n}\) es
diferenciable y \(f'(z)=nz^{n-1}\).
\end{theorem}
\end{frame}

\begin{frame}[label=sec-2-2]{Regla de la cadena}
\begin{theorem}
Sean \(f\colon A\to \mathbb{C}\) y \(g\colon B\to \mathbb{C}\)
funciones tales que \(f(A)\subseteq B\). Sean \(f\) derivable en
\(z_{0}\in A\) y \(g\) derivable en \(f(z_{0})\). Entonces
\(g\circ f\) es derivable en \(z_{0}\), y \((g\circ
    f)'(z_{0})=g'(f(z_{0}))\). 
\end{theorem}
\end{frame}

\section{Ecuaciones de Cauchy-Riemann}
\label{sec-3}

\begin{frame}[label=sec-3-1]{}
\begin{theorem}[Ecuaciones de Cauchy-Riemann]
Sea \(f\colon A\to \mathbb{C}\) derivable en \(z_{0}\). Si
\(f=u+iv\), entonces
\begin{displaymath}
u_{x}(z_{0})=v_{y}(z_{0}),\qquad u_{y}(z_{0})=-v_{x}(z_{0}).
\end{displaymath}
\end{theorem}
\end{frame}
% Emacs 24.3.1 (Org mode N/A)
\end{document}