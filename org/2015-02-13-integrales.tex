% Created 2015-02-13 vie 12:26
\documentclass[spanish,presentation]{beamer}
\usepackage[utf8]{inputenc}
\usepackage[T1]{fontenc}
\usepackage{fixltx2e}
\usepackage{graphicx}
\usepackage{longtable}
\usepackage{float}
\usepackage{wrapfig}
\usepackage{rotating}
\usepackage[normalem]{ulem}
\usepackage{amsmath}
\usepackage{textcomp}
\usepackage{marvosym}
\usepackage{wasysym}
\usepackage{amssymb}
\usepackage{capt-of}
\tolerance=1000
\usepackage{listings}
\usepackage[mono=false]{libertine}
\usepackage[scaled=0.7]{luximono}
\usepackage{lxfonts}
\usepackage[spanish, mexico, es-noshorthands, english]{babel}
% remove space between margin and lists
\usepackage{enumitem}
\setitemize{label=\usebeamerfont*{itemize item}%
\usebeamercolor[fg]{itemize item}
\usebeamertemplate{itemize item}}
\setlist{leftmargin=*,labelindent=0cm}
\setenumerate[1]{%
label=\protect\usebeamerfont{enumerate item}%
\protect\usebeamercolor[fg]{enumerate item}%
\insertenumlabel.}
\usepackage{tikz}
\usepackage{tikz-cd}
\usepackage{pgfplots}
\usetikzlibrary{babel}
\beamerdefaultoverlayspecification{<+->}
\usefonttheme{professionalfonts}
\usetheme{default}
\usecolortheme{shark}
\useinnertheme[outline,shadow]{chamfered}
\useoutertheme[glossy,nofootline]{wuerzburg}
\date{2015-02-13 7:00}
\title{Integrales}
\begin{document}

\maketitle
\setbeamertemplate{navigation symbols}{}
\setbeamertemplate{items}[circle]
\languagepath{spanish}

\tableofcontents

\section{Definiciones}
\label{sec-1}

\begin{frame}[label=sec-1-1]{}
\begin{definition}
Dado \(\gamma\colon[a,b]\to \mathbb{C}\), escribimos
\(\gamma(t)=x(t)+iy(t)\). Decimos que \(\gamma\) es continua (derivable) si
\(x,y\) son funciones continuas (derivables) \([a,b]\to
    \mathbb{R}\). 

Si \(\gamma\) es continua, definimos
\begin{displaymath}
\int_{a}^{b}\gamma(t)\,dt=\int_{a}^{b}x(t)\,dt+\int_{a}^{b}y(t)\,dt.
\end{displaymath}

Si \(\gamma\) es derivable, escribimos
\(\gamma'(t)=x'(t)+iy'(t)\). Si \(U\subseteq \mathbb{C}\) es
abierto y \(\gamma([a,b])\subseteq U\), decimos que \(\gamma\) es
una \alert{curva en \(U\)}.
\end{definition}
\end{frame}

\begin{frame}[label=sec-1-2]{}
\begin{definition}
Sean \(U\subseteq \mathbb{C}\) abierto, \(f\colon U\to
    \mathbb{C}\) una función y \(\gamma\) una curva en
\(U\). Definimos la \alert{integral de \(f\) sobre \(\gamma\)} como:
\begin{displaymath}
\int_{\gamma}f\,dz=\int_{a}^{b}f(\gamma(t))\gamma'(t)\,dt.
\end{displaymath}
\end{definition}
\end{frame}
% Emacs 24.3.1 (Org mode N/A)
\end{document}