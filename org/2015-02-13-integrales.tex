% Created 2015-02-16 lun 11:13
\documentclass[spanish,presentation]{beamer}
\usepackage[utf8]{inputenc}
\usepackage[T1]{fontenc}
\usepackage{fixltx2e}
\usepackage{graphicx}
\usepackage{longtable}
\usepackage{float}
\usepackage{wrapfig}
\usepackage{rotating}
\usepackage[normalem]{ulem}
\usepackage{amsmath}
\usepackage{textcomp}
\usepackage{marvosym}
\usepackage{wasysym}
\usepackage{amssymb}
\usepackage{capt-of}
\tolerance=1000
\usepackage{listings}
\usepackage[mono=false]{libertine}
\usepackage[scaled=0.7]{luximono}
\usepackage{lxfonts}
\usepackage[spanish, mexico, es-noshorthands, english]{babel}
% remove space between margin and lists
\usepackage{enumitem}
\setitemize{label=\usebeamerfont*{itemize item}%
\usebeamercolor[fg]{itemize item}
\usebeamertemplate{itemize item}}
\setlist{leftmargin=*,labelindent=0cm}
\setenumerate[1]{%
label=\protect\usebeamerfont{enumerate item}%
\protect\usebeamercolor[fg]{enumerate item}%
\insertenumlabel.}
\usepackage{tikz}
\usepackage{tikz-cd}
\usepackage{pgfplots}
\usetikzlibrary{babel}
\beamerdefaultoverlayspecification{<+->}
\usefonttheme{professionalfonts}
\usetheme{default}
\usecolortheme{shark}
\useinnertheme[outline,shadow]{chamfered}
\useoutertheme[glossy,nofootline]{wuerzburg}
\date{2015-02-13 7:00}
\title{Integrales}
\begin{document}

\maketitle
\setbeamertemplate{navigation symbols}{}
\setbeamertemplate{items}[circle]
\languagepath{spanish}

\tableofcontents

\section{Definiciones}
\label{sec-1}

\begin{frame}[label=sec-1-1]{}
\begin{definition}
Dado \(\gamma\colon[a,b]\to \mathbb{C}\), escribimos
\(\gamma(t)=x(t)+iy(t)\). Decimos que \(\gamma\) es continua (derivable) si
\(x,y\) son funciones continuas (derivables) \([a,b]\to
    \mathbb{R}\). Decimos que \(\gamma\) es \alert{suave} si \(\gamma(t)\) es
derivable y \(\gamma'(t)\) es continua.

Si \(\gamma\) es continua, definimos
\begin{displaymath}
\int_{a}^{b}\gamma(t)\,dt=\int_{a}^{b}x(t)\,dt+i\int_{a}^{b}y(t)\,dt.
\end{displaymath}

Si \(\gamma\) es derivable, escribimos
\(\gamma'(t)=x'(t)+iy'(t)\). Si \(U\subseteq \mathbb{C}\) es
abierto y \(\gamma([a,b])\subseteq U\), decimos que \(\gamma\) es
una \alert{curva en \(U\)}.
\end{definition}
\end{frame}

\begin{frame}[label=sec-1-2]{}
\begin{definition}
Sean \(U\subseteq \mathbb{C}\) abierto, \(f\colon U\to
    \mathbb{C}\) una función continua, y \(\gamma\) una curva suave en
\(U\). Definimos la \alert{integral de \(f\) sobre \(\gamma\)} como:
\begin{displaymath}
\int_{\gamma}f(z)\,dz=\int_{a}^{b}f(\gamma(t))\gamma'(t)\,dt.
\end{displaymath}
\end{definition}
\end{frame}

\begin{frame}[label=sec-1-3]{Propiedades de la integral}
\begin{itemize}
\item Si \(f_{1}, f_{2}\colon U\to \mathbb{C}\) y \(\gamma\) es una
curva suave en \(U\), se tiene:
\begin{displaymath}
\int_{\gamma}(f_{1}+f_{2})(z)\,dz=\int_{\gamma}f_{1}(z)\,dz+\int_{\gamma}f_{2}(z)\,dz.
\end{displaymath}
\item Si \(f\colon U\to \mathbb{C}\) y \(c\in \mathbb{C}\), se tiene:
\begin{displaymath}
\int_{\gamma}cf(z)\,dz=c\int_{\gamma}f(z)\,dz.
\end{displaymath}
\end{itemize}
\end{frame}

\begin{frame}[label=sec-1-4]{Camino inverso}
\begin{definition}[Camino inverso]
Dado \(\gamma\colon[a,b]\to \mathbb{C}\), definimos el \alert{camino
inverso} \(-\gamma\colon[a,b]\to\mathbb{C}\) como:
\begin{displaymath}
(-\gamma)(t)=\gamma(a+b-t).
\end{displaymath}
\end{definition}

\begin{block}{}
Se tiene que:
\begin{displaymath}
\int_{-\gamma}f(z)\,dz=-\int_{\gamma}f(z)\,dz.
\end{displaymath}
\end{block}
\end{frame}

\begin{frame}[label=sec-1-5]{}
\begin{definition}[Camino suave a trozos]
Sea \(U\subseteq \mathbb{C}\) abierto. Sea \(\gamma\colon [a,b]\to
    U\) continua tal que existen
\(a=x_{0}<x_{1}<\ldots<x_{n-1}<x_{n}=b\) y 
\(\gamma_{1},\gamma_{2},\ldots,\gamma_{n}\) curvas suaves en \(U\)
con dominios \([x_{i-1},x_{i}]\) para \(i=1,\ldots,n\) respectivamente.
Decimos entonces que \(\gamma\) es un \alert{camino suave a trozos} y
escribimos:
\begin{displaymath}
\gamma=\gamma_{1}+\cdots+\gamma_{n}.
\end{displaymath}
\end{definition}

\begin{definition}
Si \(\gamma\) es suave a trozos, con la notación anterior
definimos:
\begin{displaymath}
\int_{\gamma}f(z)\,dz=\int_{\gamma_{1}}f(z)\,dz+\cdots+\int_{\gamma_{n}}f(z)\,dz.
\end{displaymath}
\end{definition}
\end{frame}

\begin{frame}[label=sec-1-6]{}
\begin{theorem}
Las propiedades de la integral demostradas hasta ahora se
conservan si \(\gamma\) se supone suave a trozos.
\end{theorem}
\end{frame}
% Emacs 24.3.1 (Org mode N/A)
\end{document}