% Created 2015-02-09 lun 08:10
\documentclass[spanish,presentation]{beamer}
\usepackage[utf8]{inputenc}
\usepackage[T1]{fontenc}
\usepackage{fixltx2e}
\usepackage{graphicx}
\usepackage{longtable}
\usepackage{float}
\usepackage{wrapfig}
\usepackage{rotating}
\usepackage[normalem]{ulem}
\usepackage{amsmath}
\usepackage{textcomp}
\usepackage{marvosym}
\usepackage{wasysym}
\usepackage{amssymb}
\usepackage{capt-of}
\tolerance=1000
\usepackage{listings}
\usepackage[mono=false]{libertine}
\usepackage[scaled=0.7]{luximono}
\usepackage{lxfonts}
\usepackage[spanish, mexico, es-noshorthands, english]{babel}
% remove space between margin and lists
\usepackage{enumitem}
\setitemize{label=\usebeamerfont*{itemize item}%
\usebeamercolor[fg]{itemize item}
\usebeamertemplate{itemize item}}
\setlist{leftmargin=*,labelindent=0cm}
\setenumerate[1]{%
label=\protect\usebeamerfont{enumerate item}%
\protect\usebeamercolor[fg]{enumerate item}%
\insertenumlabel.}
\usepackage{tikz}
\usepackage{tikz-cd}
\usepackage{pgfplots}
\usetikzlibrary{babel}
\beamerdefaultoverlayspecification{<+->}
\usefonttheme{professionalfonts}
\usetheme{default}
\usecolortheme{shark}
\useinnertheme[outline,shadow]{chamfered}
\useoutertheme[glossy,nofootline]{wuerzburg}
\date{2015-02-06 9:00}
\title{Consecuencias de las ecuaciones de Cauchy-Riemann}
\begin{document}

\maketitle
\setbeamertemplate{navigation symbols}{}
\setbeamertemplate{items}[circle]
\languagepath{spanish}

\tableofcontents

\section{}
\label{sec-1}

\begin{frame}[label=sec-1-1]{}
\begin{theorem}
Sea \(f=u+iv\colon U\to \mathbb{C}\) donde \(U\) es
abierto. Supongamos que \(u_{x},u_{y},v_{x},v_{y}\) existen en
\(U\), son continuas en \(z_{0}\in U\), y satisfacen allí las
ecuaciones de Cauchy-Riemann, es decir,
\(u_{x}(z_{0})=v_{y}(z_{0})\) y
\(v_{x}(z_{0})=-u_{y}(z_{0})\). Entonces \(f\) es derivable en
\(z_{0}\), y \(f'(z_{0})=u_{x}(z_{0})+iv_{x}(z_{0})\).
\end{theorem}
\end{frame}

\begin{frame}[label=sec-1-2]{}
\begin{definition}
Si \(D\subseteq \mathbb{C}\) es abierto y conexo, decimos que
\(D\) es un \alert{dominio}.
\end{definition}

\begin{lemma}
Sea \(u\colon D\subseteq \mathbb{C}\to \mathbb{R}\) donde \(D\) es
un dominio. Si \(u_{x}(z)=u_{y}(z)=0\) para todo \(z\in D\),
entonces \(u\) es constante.
\end{lemma}
\end{frame}

\begin{frame}[label=sec-1-3]{Funciones analíticas}
\begin{definition}
Sea \(U\subseteq \mathbb{C}\) un abierto. Si \(f\) es derivable en
todo \(z\in U\), decimos que \(f\) es \alert{analítica} en \(U\).
\end{definition}

\begin{theorem}
Sea \(f\) una función analítica en un dominio \(D\subseteq
    \mathbb{C}\). Si \(f'(z)=0\) para todo \(z\in D\), entonces \(f\)
es constante.
\end{theorem}

\begin{theorem}
Sea \(f=u+iv\) analítica en un dominio \(D\subseteq
    \mathbb{C}\). Si alguna de las funciones \(u,v,|f|\) es constante
en \(D\), entonces \(f\) es constante en \(D\).
\end{theorem}
\end{frame}
% Emacs 24.3.1 (Org mode N/A)
\end{document}