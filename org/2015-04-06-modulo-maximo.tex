% Created 2015-04-09 jue 18:57
\documentclass[spanish,presentation]{beamer}
\usepackage[utf8]{inputenc}
\usepackage[T1]{fontenc}
\usepackage{fixltx2e}
\usepackage{graphicx}
\usepackage{longtable}
\usepackage{float}
\usepackage{wrapfig}
\usepackage{rotating}
\usepackage[normalem]{ulem}
\usepackage{amsmath}
\usepackage{textcomp}
\usepackage{marvosym}
\usepackage{wasysym}
\usepackage{amssymb}
\usepackage{capt-of}
\tolerance=1000
\usepackage{listings}
\usepackage[mono=false]{libertine}
\usepackage[scaled=0.7]{luximono}
\usepackage{lxfonts}
\usepackage[spanish, mexico, es-noshorthands, english]{babel}
% remove space between margin and lists
\usepackage{enumitem}
\setitemize{label=\usebeamerfont*{itemize item}%
\usebeamercolor[fg]{itemize item}
\usebeamertemplate{itemize item}}
\setlist{leftmargin=*,labelindent=0cm}
\setenumerate[1]{%
label=\protect\usebeamerfont{enumerate item}%
\protect\usebeamercolor[fg]{enumerate item}%
\insertenumlabel.}
\usepackage{tikz}
\usepackage{tikz-cd}
\usepackage{pgfplots}
\usetikzlibrary{babel}
\beamerdefaultoverlayspecification{<+->}
\usefonttheme{professionalfonts}
\usetheme{default}
\usecolortheme{shark}
\useinnertheme[outline,shadow]{chamfered}
\useoutertheme[glossy,nofootline]{wuerzburg}
\date{2015-04-06 9:00}
\title{Teoremas del módulo máximo y del mapeo abierto}
\begin{document}

\maketitle
\setbeamertemplate{navigation symbols}{}
\setbeamertemplate{items}[circle]
\languagepath{spanish}

\tableofcontents

\section{Teorema del módulo máximo}
\label{sec-1}

\begin{frame}[label=sec-1-1]{}
\begin{theorem}[Módulo máximo]
Sean \(D\) abierto y \(f\colon D\to \mathbb{C}\) analítica y no
constante. Entonces, para cada \(z_{0}\in D\) y \(\delta>0\),
existe \(z\in D(z_{0},\delta)\cap D\) tal que
\(|f(z)|>|f(z_{0})|\).
\end{theorem}
\end{frame}

\begin{frame}[label=sec-1-2]{Demostración}
\begin{itemize}
\item Si no es cierto, existen \(z_{0}\in D\) y \(\delta>0\) tal que
para todo \(z\in D(z_{0},\delta)\) se tiene que
\(|f(z)|\leq |f(z_{0})|\).
\item Sea \(r<\delta\). Tenemos que
\(f(z_{0})=\frac{1}{2\pi}\int_{0}^{2\pi}f(z_{0}+re^{i\theta})\,d\theta\),
de donde se obtiene:
\begin{displaymath}
|f(z_{0})|\leq\frac{1}{2\pi}\int_{0}^{2\pi}|f(z_{0}+re^{i\theta})|\,d\theta.
\end{displaymath}
\item Por lo que:
\begin{displaymath}
0 \leq\frac{1}{2\pi}\int_{0}^{2\pi} (|f(z_{0}+re^{i\theta})|-|f(z_{0})|)\,d\theta.
\end{displaymath}
\item Pero por nuestra hipótesis, el integrando es \(\leq 0\), por lo
que la integral es también \(\leq 0\), y por lo tanto, igual a
cero. Como el integrando es continuo, se tiene
\(|f(z_{0}+re^{i\theta})|=|f(z_{0})|\) para toda \(\theta\).
\end{itemize}
\end{frame}

\begin{frame}[label=sec-1-3]{Demostración (continuación)}
\begin{itemize}
\item Como \(r<\delta\) es arbitrario, hemos demostrado entonces que
\(|f|\) es una función constante en \(D(z_{0},\delta)\). Pero una
de las consecuencias de las ecuaciones de Cauchy-Riemman dice que
entonces \(f\) es constante en el mismo disco.
\item Se deduce entonces que \(f\) coincide con una constante en el
disco, por lo tanto es constante en todo su dominio \(D\), lo
cual contradice nuestra hipótesis de que \(f\) no es constante.
\end{itemize}
\end{frame}

\begin{frame}[label=sec-1-4]{Un corolario}
\begin{block}{Corolario}
Si \(f\) es analítica y no constante en un disco \(D(a,\delta)\) y
continua en el disco cerrado, entonces \(|f|\) alcanza su máximo
en la frontera del disco.
\end{block}
\end{frame}

\begin{frame}[label=sec-1-5]{Teorema del módulo mínimo}
\begin{theorem}[Módulo mínimo]
Sean \(D\) abierto y \(f\colon D\to \mathbb{C}\) analítica y no
constante. Entonces, para cada \(z_{0}\in D\) y \(\delta>0\),
existe \(z\in D(z_{0},\delta)\cap D\) tal que
\(|f(z)|<|f(z_{0})|\), a menos que \(|f(z_{0})|=0\).
\end{theorem}

\begin{block}{Demostración}
Si \(|f|\) alcanzara un mínimo local en \(z_{0}\ne 0\), entonces
\(g=\frac{1}{f}\) sería tal que \(|g|\) alcanza un máximo local en
\(z_{0}\), lo cual contradice que \(f\) no es constante.
\end{block}
\end{frame}


\begin{frame}[label=sec-1-6]{Lema de Schwarz}
\begin{theorem}
\begin{itemize}
\item Sea \(f\) analítica para \(|z|<1\) tal que: \(|f(z)|\leq 1\) y
\(f(0)=0\). Entonces se obtiene que \(|f(z)|\leq|z|\) para toda
\(|z|<1\) y que \(|f'(0)|\leq 1\).
\item Si además se tiene que \(|f(z)|=|z|\) para algún \(z\ne 0\), o
bien que \(|f'(0)|=1\), entonces \(f(z)=cz\) para algún \(c\)
tal que \(|c|=1\).
\end{itemize}
\end{theorem}
\end{frame}


\section{Teorema del mapeo abierto}
\label{sec-2}

\begin{frame}[label=sec-2-1]{}
\begin{theorem}[Mapeo abierto]
Si \(U\subseteq\Omega\) es abierto, y \(f\colon\Omega\to
    \mathbb{C}\) es analítica y no constante, entonces \(f(U)\subseteq
    \mathbb{C}\) es abierto. 
\end{theorem}
\end{frame}
% Emacs 24.3.1 (Org mode N/A)
\end{document}