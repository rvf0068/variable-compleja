% Created 2015-01-26 lun 13:03
\documentclass[spanish,presentation]{beamer}
\usepackage[utf8]{inputenc}
\usepackage[T1]{fontenc}
\usepackage{fixltx2e}
\usepackage{graphicx}
\usepackage{longtable}
\usepackage{float}
\usepackage{wrapfig}
\usepackage{rotating}
\usepackage[normalem]{ulem}
\usepackage{amsmath}
\usepackage{textcomp}
\usepackage{marvosym}
\usepackage{wasysym}
\usepackage{amssymb}
\usepackage{capt-of}
\tolerance=1000
\usepackage{listings}
\usepackage[mono=false]{libertine}
\usepackage[scaled=0.7]{luximono}
\usepackage{lxfonts}
\usepackage[spanish, mexico, es-noshorthands, english]{babel}
% remove space between margin and lists
\usepackage{enumitem}
\setitemize{label=\usebeamerfont*{itemize item}%
\usebeamercolor[fg]{itemize item}
\usebeamertemplate{itemize item}}
\setlist{leftmargin=*,labelindent=0cm}
\setenumerate[1]{%
label=\protect\usebeamerfont{enumerate item}%
\protect\usebeamercolor[fg]{enumerate item}%
\insertenumlabel.}
\usepackage{tikz}
\usepackage{tikz-cd}
\usepackage{pgfplots}
\usetikzlibrary{babel}
\beamerdefaultoverlayspecification{<+->}
\usefonttheme{professionalfonts}
\usetheme{default}
\usecolortheme{shark}
\useinnertheme[outline,shadow]{chamfered}
\useoutertheme[glossy,nofootline]{wuerzburg}
\date{2015-01-26 9:00}
\title{Continuidad y límites}
\begin{document}

\maketitle
\setbeamertemplate{navigation symbols}{}
\setbeamertemplate{items}[circle]
\languagepath{spanish}

\tableofcontents

\section{Funciones continuas}
\label{sec-1}

\begin{frame}[label=sec-1-1]{Continuidad}
\begin{definition}[Continuidad en un punto]
Sean \(A\subseteq \mathbb{C}\), \(a\in A\) y \(f\colon A\to
    \mathbb{C}\) una función. Decimos que \(f\) es \alert{continua en
\(a\)} si y solo si para cada \(\epsilon>0\) existe
\(\delta>0\) tal que:
\begin{displaymath}
f(A\cap D(a,\delta))\subseteq D(f(a),\epsilon).
\end{displaymath}
\end{definition}

\begin{block}{}
Decimos que \(f\colon A\to \mathbb{C}\) es \alert{continua} si es continua
en \(a\) para todo \(a\in A\).
\end{block}

\begin{theorem}
Sean \(A\subseteq \mathbb{C}\), \(a\in A\) y \(f\colon A\to
    \mathbb{C}\) una función. Entonces \(f\) es continua en \(a\)
si y solo si para toda sucesión \(z_{n}\) en \(A\) tal que
\(z_{n}\to a\), se tiene que \(f(z_{n})\to f(a)\).
\end{theorem}
\end{frame}

\begin{frame}[label=sec-1-2]{}
\begin{theorem}
Sean \(f,g\colon A\to \mathbb{C}\) funciones continuas en \(a\in
    A\). Entonces \(cf\), \(\Re f\), \(\Im f\), \(\overline{f}\),
\(f+g\), \(fg\) son continuas en \(a\). En particular si \(f,g\) son
continuas, entonces cada una de las funciones listadas son
continuas.
\end{theorem}

\begin{theorem}
Si \(g(a)\ne 0\), entonces \(\frac{f}{g}\) es continua en
\(a\). Si \(g(a)\ne 0\) para todo \(a\in A\), entonces
\(\frac{f}{g}\) es continua. 
\end{theorem}
\end{frame}

\begin{frame}[label=sec-1-3]{}
\begin{theorem}
Sean \(f\colon A\to \mathbb{C}\), \(g\colon B\to \mathbb{C}\)
funciones tales que \(f(A)\subseteq B\). Si \(f\) es continua en
\(a\in A\) y \(g\) es continua en \(f(a)\), entonces \(g\circ f\)
es continua en \(a\). En particular, si \(f\) y \(g\) son
continuas, entonces \(g\circ f\) es continua.
\end{theorem}

\begin{theorem}
Sea \(U\subseteq \mathbb{C}\) abierto. Una función \(f\colon U\to
    \mathbb{C}\) es continua si y solo si para todo abierto
\(V\subseteq \mathbb{C}\) se tiene que el conjunto
\begin{displaymath}
f^{-1}(V)=\{z\in U\mid f(z)\in V\},
\end{displaymath}
es abierto.
\end{theorem}
\end{frame}

\section{Límites}
\label{sec-2}

\begin{frame}[label=sec-2-1]{Definición de límite}
\begin{block}{}
Dados \(a\in \mathbb{C}\) y \(r>0\), denotaremos con
\(D^{*}(a,r)\) al conjunto \(D(a,r)-\{a\}\).
\end{block}

\begin{definition}[Límite]
Sean \(f\colon A\to \mathbb{C}\) una función y \(a\in \mathbb{C}\)
un punto de acumulación de \(A\). Decimos que \(c\in \mathbb{C}\)
es \alert{límite de \(f\) en \(a\)}, denotado \(\lim_{z\to a}f(z)=c\), si
para todo \(\epsilon>0\) existe \(\delta>0\) tal que:
\begin{displaymath}
f(A\cap D^{*}(a,\delta))\subseteq D(c,\epsilon).
\end{displaymath}
\end{definition}
\end{frame}

\begin{frame}[label=sec-2-2]{Relación con continuidad}
\begin{theorem}
Sea \(a\in A\) tal que \(a\) es punto de acumulación de
\(A\). Entonces \(f\colon A\to \mathbb{C}\) es continua en \(a\)
si y solo si \(\lim_{z\to a}f(z)=f(a)\).
\end{theorem}
\end{frame}

\begin{frame}[label=sec-2-3]{}
\begin{theorem}
Sea \(a\) punto de acumulación de \(A\in \mathbb{C}\). Entonces
\(f\colon A\to \mathbb{C}\) tiene límite \(c\) en \(a\) si y solo
si para toda sucesión \(z_{n}\) en \(A-\{a\}\) con límite \(a\) se
tiene que \(f(z_{n})\) converge a \(c\).
\end{theorem}

\begin{theorem}
Sea \(A\subseteq \mathbb{C}\) y \(a\) un punto de acumulación de
\(A\). Sean \(f,g\colon A\to \mathbb{C}\) funciones tales que
\(\lim_{z\to a}f(z)=l_{1}\) y \(\lim_{z\to
    a}g(z)=l_{2}\). Entonces \(cf\), \(\Re f\), \(\Im f\),
\(\overline{f}\), \(f+g\), \(fg\) tienen todas límite cuando
\(z\to a\), y su valor es \(cl_{1}\), \(\Re l_{1}\), \(\Im
    l_{1}\), \(\overline{l_{1}}\), \(l_{1}+l_{2}\) y \(l_{1}l_{2}\),
respectivamente. Si \(l_{2}\ne 0\), entonces \(\lim_{z\to
    a}\frac{f}{g}\) existe y es igual a \(\frac{l_{1}}{l_{2}}\).
\end{theorem}
\end{frame}
% Emacs 24.3.1 (Org mode N/A)
\end{document}