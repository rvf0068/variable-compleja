% Created 2015-01-22 jue 18:52
\documentclass[spanish,presentation]{beamer}
\usepackage[utf8]{inputenc}
\usepackage[T1]{fontenc}
\usepackage{fixltx2e}
\usepackage{graphicx}
\usepackage{longtable}
\usepackage{float}
\usepackage{wrapfig}
\usepackage{rotating}
\usepackage[normalem]{ulem}
\usepackage{amsmath}
\usepackage{textcomp}
\usepackage{marvosym}
\usepackage{wasysym}
\usepackage{amssymb}
\usepackage{capt-of}
\tolerance=1000
\usepackage{listings}
\usepackage[mono=false]{libertine}
\usepackage[scaled=0.7]{luximono}
\usepackage{lxfonts}
\usepackage[spanish, mexico, es-noshorthands, english]{babel}
% remove space between margin and lists
\usepackage{enumitem}
\setitemize{label=\usebeamerfont*{itemize item}%
\usebeamercolor[fg]{itemize item}
\usebeamertemplate{itemize item}}
\setlist{leftmargin=*,labelindent=0cm}
\setenumerate[1]{%
label=\protect\usebeamerfont{enumerate item}%
\protect\usebeamercolor[fg]{enumerate item}%
\insertenumlabel.}
\usepackage{tikz}
\usepackage{tikz-cd}
\usepackage{pgfplots}
\usetikzlibrary{babel}
\beamerdefaultoverlayspecification{<+->}
\usefonttheme{professionalfonts}
\usetheme{default}
\usecolortheme{shark}
\useinnertheme[outline,shadow]{chamfered}
\useoutertheme[glossy,nofootline]{wuerzburg}
\date{2015-01-22 15:00}
\title{El plano complejo}
\begin{document}

\maketitle
\setbeamertemplate{navigation symbols}{}
\setbeamertemplate{items}[circle]
\languagepath{spanish}

\tableofcontents

\section{El plano complejo}
\label{sec-1}

\begin{frame}[label=sec-1-1]{El plano complejo}
\begin{itemize}
\item Cada número complejo \(z=(x,y)=x+iy\) se puede describir en un plano
por el punto con coordenadas \(x,y\). En este \alert{plano complejo} el
eje horizontal se llama \alert{eje real} y el vertical se llama \alert{eje imaginario}.
\item En este plano, \(z\) y \(\overline{z}\) son reflejados por el eje
real. También \(|z|\) representa la distancia al origen.
\item Todo complejo distinto de cero se puede escribir como producto de
un número real y un complejo de módulo uno, a saber:
\(z=|z|\frac{z}{|z|}\).
\item Los complejos de módulo uno se dibujan en el plano complejo en la
\alert{circunferencia unitaria}: \(\{z\in \mathbb{C}\mid |z|=1\}\).
\item Si \(\theta\) es un ángulo tal que \(\cos\theta=\frac{x}{|z|}\) y
\(\sin\theta=\frac{y}{|z|}\), entonces el complejo \(z\ne 0\) se
escribe como:
\begin{displaymath}
z=|z|(\cos \theta+i\sin\theta)
\end{displaymath}
llamada \alert{forma polar} del número complejo.
\end{itemize}
\end{frame}

\section{Argumento de un número complejo.}
\label{sec-2}

\begin{frame}[label=sec-2-1]{Puntos en la circunferencia unitaria}
\begin{figure}[htb]
\centering
\begin{tikzpicture}
  \draw[->] (-3,0) -- (3,0);
  \draw[->] (0,-3) -- (0,3);
  \draw[line width=1.5pt] (0,0) circle (2cm);
  \draw[fill,line width=1.5pt] (60:2) circle (2pt);
  \node[label={\(z=\cos\theta+i\sin\theta\)},right=1cm] at (60:2) {};
  \draw[blue,line width=1pt] (0,0) -- (60:2);
  \draw[red,line width=1pt] (1,0) arc (0:60:1);
  \node[label={\color{red}\(\theta=\arg z\)},right=0.6cm] at (10:1) {};
\end{tikzpicture}
\end{figure}


Cualquier ángulo \(\theta\) que cumpla que
\(z=\cos\theta+i\sin\theta\) se llama \alert{argumento} de \(z\).
\end{frame}

\begin{frame}[label=sec-2-2]{Ejemplos}
Para un complejo \(z\ne 0\), definimos su \alert{argumento} \(\arg z\) como
el argumento de \(\frac{z}{|z|}\).

Por ejemplo:

\begin{itemize}
\item \(\arg i=90^{\circ}=\frac{\pi}{2}\),
\item \(\arg(i+1)=\frac{\pi}{4}\),
\item \(\arg(-1)=\pi\).
\item También \(3\pi\), \(-\pi\), etc. son valores de \(\arg(-1)\).
\end{itemize}
\end{frame}

\begin{frame}[label=sec-2-3]{Propiedades del argumento}
\begin{itemize}
\item Si \(z_{1}=\cos\theta+i\sin\theta\) y
\(z_{2}=\cos\psi+i\sin\psi\), entonces 
\begin{displaymath}
z_{1}z_{2}=\cos(\theta+\psi)+i\sin(\theta+\psi).
\end{displaymath}
\item Por lo tanto, se obtiene la \alert{fórmula de DeMoivre}:
\begin{displaymath}
(\cos\theta+i\sin\theta)^{n}=\cos n\theta+i\sin n\theta
\end{displaymath}
\item La fórmula de DeMoivre se puede usar además para extraer raíces a
números complejos.
\end{itemize}
\end{frame}
% Emacs 24.3.1 (Org mode N/A)
\end{document}