% Created 2015-04-21 mar 17:54
\documentclass[spanish,presentation]{beamer}
\usepackage[utf8]{inputenc}
\usepackage[T1]{fontenc}
\usepackage{fixltx2e}
\usepackage{graphicx}
\usepackage{longtable}
\usepackage{float}
\usepackage{wrapfig}
\usepackage{rotating}
\usepackage[normalem]{ulem}
\usepackage{amsmath}
\usepackage{textcomp}
\usepackage{marvosym}
\usepackage{wasysym}
\usepackage{amssymb}
\usepackage{capt-of}
\tolerance=1000
\usepackage{listings}
\usepackage[mono=false]{libertine}
\usepackage[scaled=0.7]{luximono}
\usepackage{lxfonts}
\usepackage[spanish, mexico, es-noshorthands, english]{babel}
% remove space between margin and lists
\usepackage{enumitem}
\setitemize{label=\usebeamerfont*{itemize item}%
\usebeamercolor[fg]{itemize item}
\usebeamertemplate{itemize item}}
\setlist{leftmargin=*,labelindent=0cm}
\setenumerate[1]{%
label=\protect\usebeamerfont{enumerate item}%
\protect\usebeamercolor[fg]{enumerate item}%
\insertenumlabel.}
\usepackage{tikz}
\usepackage{tikz-cd}
\usepackage{pgfplots}
\usetikzlibrary{babel}
\beamerdefaultoverlayspecification{<+->}
\usefonttheme{professionalfonts}
\usetheme{default}
\usecolortheme{shark}
\useinnertheme[outline,shadow]{chamfered}
\useoutertheme[glossy,nofootline]{wuerzburg}
\date{2015-04-20 9:00}
\title{Ejemplos de series}
\begin{document}

\maketitle
\setbeamertemplate{navigation symbols}{}
\setbeamertemplate{items}[circle]
\languagepath{spanish}

\tableofcontents

\section{Ejemplos}
\label{sec-1}

\begin{frame}[label=sec-1-1]{Ejemplo}
\begin{itemize}
\item \(g(z)=\sum_{n=1}^{\infty}\frac{z^{n}}{n}\) converge uniformemente en
\(D_{r}=\overline{D}(0,r)\) para cada \(r<1\).
\item \(g(z)\) converge puntualmente en \(D(0,1)\).
\item \(g(z)\) no converge uniformemente en \(D(0,1)\).
\end{itemize}
\end{frame}

\begin{frame}[label=sec-1-2]{Ejemplo}
\begin{itemize}
\item \(\sum_{n=0}^{\infty}z^{n}\) converge puntualmente a
\(f(z)=\frac{1}{1-z}\) en \(D=D(0,1)\).
\item La convergencia es absoluta en \(D_{r}=\overline{D}(0,r)\) para
cada \(r<1\).
\item \(\sum_{n=1}^{\infty}nz^{n-1}=\sum_{n=0}^{\infty}(n+1)z^{n}\)
converge puntualmente en \(D\) a \(\frac{1}{(1-z)^{2}}\) y
absolutamente en \(D_{r}=\overline{D}(0,r)\) para cada \(r<1\).
\end{itemize}
\end{frame}

\begin{frame}[label=sec-1-3]{Logaritmo}
\begin{exampleblock}{}
\(\sum_{n=1}^{\infty}(-1)^{n-1}\frac{z^{n}}{n}\) converge
 uniformemente a \(\log(1+z)\) en todo disco cerrado centrado en
 el origen contenido en \(D=D(0,1)\).
\end{exampleblock}

\begin{itemize}
\item Sabemos que si \(w=\rho e^{i\theta}\in D(1,1)\), entonces
\(\mathrm{Log}(w)=\int_{\gamma}\frac{1}{\xi}d\xi=\log
      \rho+i\theta\), donde \(\mathrm{Log}\) es la rama principal del
logaritmo y \(\gamma\) es cualquier camino contenido en
\(D(1,1)\) que va de \(1\) a \(w\).
\item Haciendo el cambio de variable \(\xi=\eta+1\), obtenemos que 
\begin{displaymath}
\int_{\gamma}\frac{1}{\xi}d\xi=\int_{\beta}\frac{1}{\eta+1}d\eta, 
\end{displaymath}
donde \(\beta=\gamma-1\) es un camino de \(0\) a \(w-1\).
\end{itemize}
\end{frame}

\begin{frame}[label=sec-1-4]{}
\begin{itemize}
\item Tenemos que \(\frac{1}{\eta+1}=1-\eta+\eta^{2}-\eta^{3}+\cdots\)
uniformemente en discos cerrados contenidos en \(D(0,1)\), por lo
que:
\begin{displaymath}
\mathrm{Log}(w)=\int_{\beta}\frac{1}{\eta+1}d\eta =
\int_{\beta}(1-\eta+\eta^{2}-\eta^{3}+\cdots)\,d\eta
\end{displaymath}
\item Lo último es igual a 
\begin{displaymath}
\sum_{k=0}^{\infty}(-1)^{k}\int_{\beta}\eta^{k}\,d\eta=
\sum_{k=0}^{\infty}(-1)^{k}\left.\frac{\eta^{k+1}}{k+1}\right|_{0}^{w-1}.
\end{displaymath}
\item Lo anterior, resulta ser igual a
\(\sum_{k=1}^{\infty}(-1)^{k-1}\frac{(w-1)^{k}}{k}\). Sustituyendo
\(w-1=z\), resulta:
\begin{displaymath}
\mathrm{Log}(1+z)=\sum_{n=1}^{\infty}(-1)^{n-1}\frac{z^{n}}{n}.
\end{displaymath}
y como en el primer ejemplo, se demuestra la convergencia
uniforme en \(\overline{D_{r}}\).
\end{itemize}
\end{frame}

\begin{frame}[label=sec-1-5]{Función \(\zeta\) de Riemann}
\begin{itemize}
\item La función
\begin{displaymath}
\zeta(z)=\sum_{n=1}^{\infty}\frac{1}{n^{z}}
\end{displaymath}
es analítica en \(A=\{z\mid\Re z>1\}\).
\item Tenemos que 
\begin{displaymath}
|n^{-z}|=|e^{-z\log n}|=|e^{-x\log n-iy\log n}|=e^{-x\log n}=n^{-x}.
\end{displaymath}
\item Sea \(B\subseteq A\) un disco cerrado. Sea \(\delta\) la
distancia entre \(B\) y el complemento de \(A\). Entonces, si
\(z\in B\), se tiene que \(x\geq 1+\delta\), por lo que
\(n^{-x}\leq n^{-(1+\delta)}\).
\item Tomando \(M_{n}=n^{-(1+\delta)}\), se obtiene la convergencia
uniforme de la serie en \(B\).
\end{itemize}
\end{frame}
% Emacs 24.3.1 (Org mode N/A)
\end{document}