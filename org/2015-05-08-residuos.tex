% Created 2015-05-07 jue 18:05
\documentclass[spanish,presentation]{beamer}
\usepackage[utf8]{inputenc}
\usepackage[T1]{fontenc}
\usepackage{fixltx2e}
\usepackage{graphicx}
\usepackage{longtable}
\usepackage{float}
\usepackage{wrapfig}
\usepackage{rotating}
\usepackage[normalem]{ulem}
\usepackage{amsmath}
\usepackage{textcomp}
\usepackage{marvosym}
\usepackage{wasysym}
\usepackage{amssymb}
\usepackage{capt-of}
\tolerance=1000
\usepackage{listings}
\usepackage[mono=false]{libertine}
\usepackage[scaled=0.7]{luximono}
\usepackage{lxfonts}
\usepackage[spanish, mexico, es-noshorthands, english]{babel}
% remove space between margin and lists
\usepackage{enumitem}
\setitemize{label=\usebeamerfont*{itemize item}%
\usebeamercolor[fg]{itemize item}
\usebeamertemplate{itemize item}}
\setlist{leftmargin=*,labelindent=0cm}
\setenumerate[1]{%
label=\protect\usebeamerfont{enumerate item}%
\protect\usebeamercolor[fg]{enumerate item}%
\insertenumlabel.}
\usepackage{tikz}
\usepackage{tikz-cd}
\usepackage{pgfplots}
\usetikzlibrary{babel}
\usetheme{default}
\usecolortheme{shark}
\useinnertheme[outline,shadow]{chamfered}
\useoutertheme[glossy,nofootline]{wuerzburg}
\date{2015-05-08 7:00}
\title{Residuos y cálculo de integrales}
\beamerdefaultoverlayspecification{<+->}
\usefonttheme{professionalfonts}
\hypersetup{
 pdfauthor={},
 pdftitle={Residuos y cálculo de integrales},
 pdfkeywords={},
 pdfsubject={},
 pdfcreator={Emacs 24.3.1 (Org mode N/A)}, 
 pdflang={English}}
\begin{document}

\maketitle
\setbeamertemplate{navigation symbols}{}
\setbeamertemplate{items}[circle]
\languagepath{spanish}

\tableofcontents

\section{Definición}
\label{orgheadline1}

\begin{frame}[label=sec-1-1]{}
\begin{definition}[Residuo]
Supongamos que \(f\) tiene una singularidad aislada en
\(z_{0}\). Si la expansión de Laurent alrededor de \(z_{0}\) es:
\begin{displaymath}
\cdots+\frac{b_{2}}{(z-z_{0})^{2}}+\frac{b_{1}}{(z-z_{0})}+a_{0}+a_{1}(z-z_{0})+\cdots
\end{displaymath}
entonces \(b_{1}\) se llama el \alert{residuo} de \(f\) en \(z_{0}\), y
se denota como
\begin{displaymath}
b_{1}=\mathrm{Res}(f,z_{0}).
\end{displaymath}
\end{definition}

\begin{block}{Observación}
Si \(\lim_{z\to z_{0}}(z-z_{0})f(z)\) existe, es igual a
\(\mathrm{Res}(f,z_{0})\). En tal caso, \(f(z)\) tiene una
singularidad removible en \(z_{0}\), o un polo de orden 1
(\emph{simple}). 
\end{block}
\end{frame}

\begin{frame}[label=sec-1-2]{}
\begin{theorem}[Cálculo de residuos]
Sean \(g,h\) analíticas en \(z_{0}\), y supongamos que
\(g(z_{0})\ne 0\), \(h(z_{0})=0\), y \(h'(z_{0})\ne 0\). Entonces
\(f(z)=\frac{g(z)}{h(z)}\) tiene un polo simple en \(z_{0}\), y
\begin{displaymath}
\mathrm{Res}(f,z_{0})=\frac{g(z_{0})}{h'(z_{0})}
\end{displaymath}
\end{theorem}

\begin{theorem}[Generalización]
Supongamos que \(g\) tiene un cero de orden \(k\) en \(z_{0}\) y
que \(h\) tiene un cero de orden \(k+1\) en \(z_{0}\). Entonces
\(f(z)=\frac{g(z)}{h(z)}\) tiene un polo simple en \(z_{0}\), y
\begin{displaymath}
\mathrm{Res}(f,z_{0})=(k+1)\frac{g^{(k)}(z_{0})}{h^{(k+1)}(z_{0})}
\end{displaymath}
\end{theorem}
\end{frame}

\begin{frame}[label=sec-1-3]{}
\begin{theorem}[Residuo en polo de orden dos]
Sean \(g,h\) analíticas en \(z_{0}\), y supongamos que
\(g(z_{0})\ne 0\), \(h(z_{0})=h'(z_{0})=0\) y \(h''(z_{0})\ne
    0\). Entonces \(f(z)=\frac{g(z)}{h(z)}\) tiene un polo de orden dos en
\(z_{0}\), y
\begin{displaymath}
\mathrm{Res}(f,z_{0})=2\frac{g'(z_{0})}{h''(z_{0})}-\frac{2}{3}\frac{g(z_{0})h'''(z_{0})}{[h''(z_{0})]^2}
\end{displaymath}    
\end{theorem}
\end{frame}

\begin{frame}[label=sec-1-4]{}
\begin{theorem}[Teorema del residuo]
Sea \(D\subseteq \mathbb{C}\) un dominio estrellado. Sean
\(z_{1}, z_{2},\ldots,z_{n}\in D\). Sea \(f\) una función
analítica en \(D-\{z_{1}, z_{2},\ldots,z_{n}\}\). Sea \(\gamma\)
una curva cerrada en \(D\). Entonces:
\begin{displaymath}
\int_{\gamma}f(z)\,dz=2\pi i\sum_{i=1}^{n}[\mathrm{Res}(f,z_{0})n(f,z_{0})].
\end{displaymath}
\end{theorem}
\end{frame}

\section{Cálculo de integrales reales}
\label{orgheadline1}

\begin{frame}[label=sec-2-1]{}
\begin{theorem}[Cálculo de integral impropia]
Sea \(f\) analítica en \(\mathbb{C}\), salvo por una cantidad
finita de polos, ninguno en el eje real. Supongamos que existen
\(M,R\) tales que:
\begin{displaymath}
|f(z)| \leq \frac{M}{|z|^2}.
\end{displaymath}
para \(|z|\geq R\). Entonces \(\int_{-\infty}^{\infty}f(x)\,dx\)
es igual a:
\begin{displaymath}
2\pi i\sum [\text{residuos de \(f\) en el semiplano superior}]
\end{displaymath}
\end{theorem}

\begin{block}{Observación}
Las hipótesis del teorema anterior se cumplen para
\(f=\frac{P}{Q}\), si \(P,Q\) son polinomios, el grado de \(Q\) es
mayor que \(2+\mathrm{grado}(P)\), y \(Q\) no tiene ceros en el
eje real.
\end{block}
\end{frame}
\end{document}