% Created 2015-01-19 lun 08:46
\documentclass[spanish,presentation]{beamer}
\usepackage[utf8]{inputenc}
\usepackage[T1]{fontenc}
\usepackage{fixltx2e}
\usepackage{graphicx}
\usepackage{longtable}
\usepackage{float}
\usepackage{wrapfig}
\usepackage{rotating}
\usepackage[normalem]{ulem}
\usepackage{amsmath}
\usepackage{textcomp}
\usepackage{marvosym}
\usepackage{wasysym}
\usepackage{amssymb}
\usepackage{capt-of}
\tolerance=1000
\usepackage{listings}
\usepackage[mono=false]{libertine}
\usepackage[scaled=0.7]{luximono}
\usepackage{lxfonts}
\usepackage[spanish, mexico, es-noshorthands, english]{babel}
% remove space between margin and lists
\usepackage{enumitem}
\setitemize{label=\usebeamerfont*{itemize item}%
\usebeamercolor[fg]{itemize item}
\usebeamertemplate{itemize item}}
\setlist{leftmargin=*,labelindent=0cm}
\setenumerate[1]{%
label=\protect\usebeamerfont{enumerate item}%
\protect\usebeamercolor[fg]{enumerate item}%
\insertenumlabel.}
\usepackage{tikz}
\usepackage{tikz-cd}
\usepackage{pgfplots}
\usetikzlibrary{babel}
\beamerdefaultoverlayspecification{<+->}
\usefonttheme{professionalfonts}
\usetheme{default}
\usecolortheme{shark}
\useinnertheme[outline,shadow]{chamfered}
\useoutertheme[glossy,nofootline]{wuerzburg}
\date{2015-01-19 9:00}
\title{Los números complejos}
\begin{document}

\maketitle
\setbeamertemplate{navigation symbols}{}
\setbeamertemplate{items}[circle]
\languagepath{spanish}

\tableofcontents

\section{Campos}
\label{sec-1}

\begin{frame}[label=sec-1-1]{Definición de campo}
Un campo es una estructura formada por un conjunto \(F\) y dos
operaciones binarias \(F\times F\to F\), la primera llamada \alert{suma}
y denotada por \((a,b)\mapsto a+b\), la segunda llamada \alert{producto} y
denotada por \((a,b)\mapsto ab\), tales que:

\begin{itemize}
\item \(a+b=b+a\) para todos \(a,b\in F\),
\item \(a+(b+c)=(a+b)+c\) para todos \(a,b,c\in F\),
\item existe un elemento \(0\in F\) tal que \(a+0=a\) para todo \(a\in F\),
\item para todo \(a\in F\) existe \(-a\in F\) tal que \(a+(-a)=0\),
\item \(ab=ba\) para todos \(a,b\in F\),
\item \(a(bc)=(ab)c\) para todos \(a,b,c\in F\),
\item existe un elemento \(1\in F\) con \(1\ne 0\) tal que \(a1=a\) para todo \(a\in F\),
\item para todo \(a\in F\), \(a\ne 0\) existe \(a^{-1}\in F\) tal que \(a(a^{-1})=1\),
\item \(a(b+c)=ab+ac\) para todos \(a,b,c\in F\).
\end{itemize}
\end{frame}

\begin{frame}[label=sec-1-2]{Ejemplos de campos}
\begin{itemize}
\item El conjunto de los números racionales \(\mathbb{Q}\), con las
operaciones usuales.
\item El conjunto de los números reales \(\mathbb{R}\), con las
operaciones usuales.
\item Existen campos con una cantidad finita de elementos. (De hecho,
existe un campo con \(n\) elementos si y solo si \(n=p^{r}\) para
\(p\) primo y \(r>0\).)
\item El campo de los números complejos, que estudiaremos aquí.
\end{itemize}
\end{frame}

\begin{frame}[label=sec-1-3]{Observaciones}
\begin{itemize}
\item En todo campo se define \(a-b\) como \(a+(-b)\). Si \(b\ne 0\) se
define \(\frac{a}{b}\) como \(ab^{-1}\).
\item Existen muchas propiedades que se deducen solamente a partir de los
axiomas de campo. Por ejemplo, se tiene que los elementos \(0,1\)
son únicos con respecto a las propiedades que los definen, y que
\(a0=0\) para todo elemento del campo \(a\).
\end{itemize}
\end{frame}

\section{Caracterización de los números reales}
\label{sec-2}

\begin{frame}[label=sec-2-1]{Campos ordenados}
\begin{definition}[Campo ordenado]
Decimos que \(F\) es un \alert{campo ordenado} si existe un conjunto
\(P\subseteq F\) tal que:

\begin{itemize}
\item \(a+b\in P\) para todos \(a,b\in P\),
\item \(ab\in P\) para todos \(a,b\in P\),
\item \(F\) es unión disjunta de \(\{0\}\), \(P\), y \(\{-a\mid a\in P\}\).
\end{itemize}
\end{definition}

\begin{exampleblock}{Ejemplo}
Los campos \(\mathbb{Q}\) y \(\mathbb{R}\) son ordenados.
\end{exampleblock}
\end{frame}

\begin{frame}[label=sec-2-2]{Campos ordenados completos}
\begin{block}{Relación de orden}
En un campo ordenado, se puede definir la relación \(a>b\) como
\(a-b\in P\).
\end{block}

\begin{definition}[Campo ordenado completo]
Si \(F\) es un campo ordenado donde todo subconjunto no vacío
acotado superiormente tiene una mínima cota superior, decimos que
\(F\) es \alert{completo}.
\end{definition}

\begin{theorem}[Caracterización de \(\mathbb{R}\)]
Salvo isomorfismo, el único campo ordenado completo es el campo de
los números reales.
\end{theorem}
\end{frame}

\section{Números complejos}
\label{sec-3}

\begin{frame}[label=sec-3-1]{Definición de Hamilton (1833)}
Sea \(\mathbb{C}=\{(x,y)\mid x,y\in \mathbb{R}\}\). Entonces, en
\(\mathbb{C}\) podemos definir operaciones de suma y producto:

\begin{itemize}
\item \((x,y)+(u,v)=(x+u,y+v)\),
\item \((x,y)(u,v)=(xu-yv,xv+yu)\),
\end{itemize}

de tal modo que \(\mathbb{C}\) resulta ser un campo.
\end{frame}

\begin{frame}[label=sec-3-2]{Propiedades de \(\mathbb{C}\)}
\begin{itemize}
\item En \(\mathbb{C}\) tenemos \(0=(0,0)\) y \(1=(1,0)\).
\item ¿Cuál es el inverso multiplicativo de \((x,y)\ne 0\)?
\item El subconjunto \(\{(x,0)\in \mathbb{C}\mid x\in \mathbb{R}\}\)
es cerrado bajo las operaciones definidas en \(\mathbb{C}\), y
resulta ser un campo isomorfo a \(\mathbb{R}\) bajo la
correspondencia \(x\leftrightarrow (x,0)\).
\item Por lo anterior, denotaremos a \((x,0)\) por \(x\).
\item Tenemos que \((0,y)=y(0,1)\) para todo \(y\in
     \mathbb{R}\). Denotaremos a \((0,1)\) por \(i\).
\item Se tiene entonces que \(i^{2}=(0,1)(0,1)=(-1,0)=-1.\)
\item Además, \((0,y)=(0,1)(y,0)=iy\) para todo \(y\in\mathbb{R}\). Por
lo tanto, \((x,y)=(x,0)+(0,y)=x+iy\) para todo
\((x,y)\in\mathbb{C}\).
\end{itemize}
\end{frame}

\begin{frame}[label=sec-3-3]{Breviario cultural}
\begin{itemize}
\item ¿Es posible definir una estructura de campo en el conjunto de
tercias de números reales? ¿O en general en \(\mathbb{R}^{n}\)?
\item Hamilton no lo logró en \(\mathbb{R}^{3}\). Pero en 1843 definió
una estructura (\(\mathbb{H}\)) de \alert{álgebra con división} en
\(\mathbb{R}^{4}\), la cual cumple los axiomas de campo excepto
la conmutatividad del producto.
\item Frobenius probó en 1877 que las únicas álgebras con división de
dimensión finita sobre \(\mathbb{R}\) son: \(\mathbb{R}\),
\(\mathbb{C}\) y \(\mathbb{H}\).
\item Usando que \(\mathbb{C}\) es \alert{algebraicamente cerrado} (es decir,
todo polinomio con coeficientes en \(\mathbb{C}\) tiene una raíz
en \(\mathbb{C}\)), es fácil demostrar que el único campo que
extiende a \(\mathbb{C}\) y es de dimensión finita como espacio
vectorial sobre \(\mathbb{R}\) es \(\mathbb{C}\).
\end{itemize}
\end{frame}
\section{Definiciones}
\label{sec-4}

\begin{frame}[label=sec-4-1]{Definiciones}
\begin{itemize}
\item El número complejo \(z=(x,y)\) se denotará como \(z=x+iy\). Decimos
que \(x\) es la \alert{parte real} de \(z\) y que \(y\) es la \alert{parte
imaginaria} de \(z\). Escribimos \(x=\Re z\), \(y=\Im z\).
\item El \alert{conjugado} de \(z=x+iy\) es \(\overline{z}=x-iy\).
\item La \alert{magnitud} de \(z=x+iy\) es \(|z|=\sqrt{x^{2}+y^{2}}\).
\end{itemize}
\end{frame}

\begin{frame}[label=sec-4-2]{Propiedades}
\begin{itemize}
\item \(\overline{\overline{z}}=z\),
\(\overline{z+w}=\overline{z}+\overline{w}\),
\(\overline{zw}=\overline{z}\overline{w}\),
\(\overline{\frac{z}{w}}=\frac{\overline{z}}{\overline{w}}\).
\item \(\Re z =\frac{z+\overline{z}}{2}\), \(\Im z =\frac{z-\overline{z}}{2}\).
\item \(|zw|=|z||w|\), \(|\frac{z}{w}|=\frac{|z|}{|w|}\).
\item \(|z|=|\overline{z}|\), \(z\overline{z}=|z|^{2}\).
\item \(|\Re z|\leq |z|\), \(|\Im z|\leq |z|\).
\item \(|z+w|\leq |z|+|w|\), \(|z+w|\geq ||z|-|w||\).
\item Si \(z\ne 0\), \(z^{-1}=\frac{\overline{z}}{|z|^{2}}\).
\end{itemize}
\end{frame}
% Emacs 24.3.1 (Org mode N/A)
\end{document}