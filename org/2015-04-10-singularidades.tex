% Created 2015-04-09 jue 18:57
\documentclass[spanish,presentation]{beamer}
\usepackage[utf8]{inputenc}
\usepackage[T1]{fontenc}
\usepackage{fixltx2e}
\usepackage{graphicx}
\usepackage{longtable}
\usepackage{float}
\usepackage{wrapfig}
\usepackage{rotating}
\usepackage[normalem]{ulem}
\usepackage{amsmath}
\usepackage{textcomp}
\usepackage{marvosym}
\usepackage{wasysym}
\usepackage{amssymb}
\usepackage{capt-of}
\tolerance=1000
\usepackage{listings}
\usepackage[mono=false]{libertine}
\usepackage[scaled=0.7]{luximono}
\usepackage{lxfonts}
\usepackage[spanish, mexico, es-noshorthands, english]{babel}
% remove space between margin and lists
\usepackage{enumitem}
\setitemize{label=\usebeamerfont*{itemize item}%
\usebeamercolor[fg]{itemize item}
\usebeamertemplate{itemize item}}
\setlist{leftmargin=*,labelindent=0cm}
\setenumerate[1]{%
label=\protect\usebeamerfont{enumerate item}%
\protect\usebeamercolor[fg]{enumerate item}%
\insertenumlabel.}
\usepackage{tikz}
\usepackage{tikz-cd}
\usepackage{pgfplots}
\usetikzlibrary{babel}
\beamerdefaultoverlayspecification{<+->}
\usefonttheme{professionalfonts}
\usetheme{default}
\usecolortheme{shark}
\useinnertheme[outline,shadow]{chamfered}
\useoutertheme[glossy,nofootline]{wuerzburg}
\date{2015-04-10 7:00}
\title{Clasificación de singularidades}
\begin{document}

\maketitle
\setbeamertemplate{navigation symbols}{}
\setbeamertemplate{items}[circle]
\languagepath{spanish}

\tableofcontents

\section{Definición}
\label{sec-1}

\begin{frame}[label=sec-1-1]{Singularidades aisladas}
\begin{definition}[Singularidad]
Si \(a\in \mathbb{C}\) es tal que la función compleja \(f\) es
analítica en \(\{z\in \mathbb{C}\mid 0<|z-a|<\delta\}\) para algún
\(\delta>0\), decimos que \(a\) es \alert{singularidad (aislada)} de
\(f\).
\end{definition}

\begin{block}{}
Si \(\lim_{z\to a}(z-a)f(z)=0\), entonces podemos definir el valor
de \(f\) en \(a\), de modo que \(f\) es analítica en toda una
vecindad de \(a\). En tal caso, se dice que \(a\) es \alert{singularidad
removible}. 
\end{block}
\end{frame}

\begin{frame}[label=sec-1-2]{Polos}
\begin{definition}[Polo]
Si \(a\) es una singularidad aislada, y \(\lim_{z\to
    a}f(z)=\infty\), decimos que \(a\) es \alert{polo} de \(f\).
\end{definition}

\begin{exampleblock}{Ejemplo}
Por ejemplo, \(f(z)=\frac{1}{(z+i)^{2}}\) tiene un polo en
\(z=-i\). 
\end{exampleblock}
\end{frame}

\begin{frame}[label=sec-1-3]{}
\begin{block}{Observaciones}
\begin{itemize}
\item Si \(a\) es polo, existe \(\delta'\leq\delta\) tal que \(f(z)\ne
      0\) para \(z\) tal que \(0<|z-a|<\delta'\).
\item En tal dominio, la función \(g(z)=\frac{1}{f(z)}\) está definida
y es analítica, más aún, \(a\) es singularidad removible y \(g\)
se puede definir como \(g(a)=0\).
\item Como \(g\) no es idénticamente cero, podemos suponer que el cero
\(a\) tiene un orden \(h\), es decir,
\(g(z)=(z-a)^{h}g_{h}(z)\), donde \(g_{h}(a)\ne 0\).
\item En tal caso, 
\begin{displaymath}
f(z)=\frac{f_{h}(z)}{(z-a)^{h}},
\end{displaymath}
donde \(f_{h}(z)=\frac{1}{g_{h}(z)}\).
\end{itemize}
\end{block}
\end{frame}

\begin{frame}[label=sec-1-4]{Funciones meromorfas}
\begin{definition}
Si \(f\) es analítica en una región \(\Omega\), salvo por polos,
decimos que \(f\) es \alert{meromorfa} en \(\Omega\).
\end{definition}

\begin{block}{Observación}
La suma, producto y cociente de funciones meromorfas es
meromorfas, siempre que el divisor de un cociente no sea la
función idénticamente cero. 
\end{block}
\end{frame}

\section{Clasificación de singularidades}
\label{sec-2}

\begin{frame}[label=sec-2-1]{Condiciones}
Consideremos las siguientes propiedades acerca de la función \(f\)
con una singularidad aislada \(a\), y \(\alpha\in \mathbb{R}\):

\begin{description}
\item[{Condición 1:}] \(\lim_{z\to a}|z-a|^{\alpha}|f(z)|=0\),
\item[{Condición 2:}] \(\lim_{z\to a}|z-a|^{\alpha}|f(z)|=\infty\).
\end{description}
\end{frame}

\begin{frame}[label=sec-2-2]{}
\begin{itemize}
\item Supongamos que la condición 1 se cumple para cierta \(\alpha\).
\item Entonces, también se cumple para toda \(\alpha'>\alpha\), y por
lo tanto, para algún entero \(m\).
\item En tal caso, \((z-a)^{m}f(z)\) tiene una singularidad removible
en \(a\), si \(f(z)\) no es idénticamente cero, \(a\) es un cero,
digamos de orden \(k\).
\item Entonces \((z-a)^{m}f(z)=(z-a)^{k}f_{k}(z)\). Escribimos
\begin{displaymath}
(z-a)^{m-k}f(z)=f_{k}(z),
\end{displaymath}
de donde se obtiene que, si \(\alpha>m-k\), se cumple la condición
1, y si \(\alpha<m-k\), se cumple la condición 2.
\end{itemize}
\end{frame}

\begin{frame}[label=sec-2-3]{}
\begin{itemize}
\item Supongamos ahora que la condición 2 se cumple para cierta \(\alpha\).
\item Entonces, también se cumple para toda \(\alpha'<\alpha\), y por
lo tanto, para algún entero \(n\).
\item En tal caso, \((z-a)^{n}f(z)\) tiene un polo en \(a\), digamos
de orden \(l\), es decir \((z-a)^{n}f(z)=\frac{f_{l}(z)}{(z-a)^{l}}\).
\item Podemos escribir entonces
\begin{displaymath}
(z-a)^{n+l}f(z)=f_{l}(z).
\end{displaymath}
\item De lo anterior, se obiente que si \(\alpha>n+l\), se cumple la condición
1, y si \(\alpha<n+l\), se cumple la condición 2.
\end{itemize}
\end{frame}

\begin{frame}[label=sec-2-4]{Singularidades esenciales}
\begin{definition}[Singularidad esencial]
Si \(a\) es una singularidad aislada tal que no se cumple la
condición 1 ni la condición 2 para ninguna \(\alpha\in
    \mathbb{R}\), decimos que \(a\) es \alert{singularidad esencial} de \(f\).
\end{definition}

\begin{theorem}[Casorati–Weierstrass]
Si \(a\) es una singularidad esencial de \(f\), entonces para toda
\(0<\delta'<\delta\), se tiene que \(f(D(a,\delta'))\) es denso en
\(\mathbb{C}\). 
\end{theorem}
\end{frame}

\begin{frame}[label=sec-2-5]{Demostración}
\begin{itemize}
\item Si el teorema no fuera cierto, existirían un número complejo
\(A\) y un \(r>0\) tal que \(|f(z)-A|>r\) para todo \(z\) en
alguna vecindad perforada de \(a\).
\item Para \(\alpha<0\), entonces \(\lim_{z\to
     a}|z-a|^{\alpha}|f(z)-A|=\infty\), lo cual implica que \(a\) no
es singularidad esencial de \(f(z)-A\).
\item Existe \(\beta>0\) tal que \(\lim_{z\to
     a}|z-a|^{\beta}|f(z)-A|=0\), además \(\lim_{z\to a}|z-a|^{\beta}|A|=0\).
\item De lo anterior, se obtiene que
\begin{displaymath}
\lim_{z\to a}|z-a|^{\beta}|f(z)|=0,
\end{displaymath}
lo cual contradice que \(a\) es singularidad esencial.
\end{itemize}
\end{frame}
% Emacs 24.3.1 (Org mode N/A)
\end{document}