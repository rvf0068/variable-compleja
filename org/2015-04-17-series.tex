% Created 2015-04-16 jue 18:46
\documentclass[spanish,presentation]{beamer}
\usepackage[utf8]{inputenc}
\usepackage[T1]{fontenc}
\usepackage{fixltx2e}
\usepackage{graphicx}
\usepackage{longtable}
\usepackage{float}
\usepackage{wrapfig}
\usepackage{rotating}
\usepackage[normalem]{ulem}
\usepackage{amsmath}
\usepackage{textcomp}
\usepackage{marvosym}
\usepackage{wasysym}
\usepackage{amssymb}
\usepackage{capt-of}
\tolerance=1000
\usepackage{listings}
\usepackage[mono=false]{libertine}
\usepackage[scaled=0.7]{luximono}
\usepackage{lxfonts}
\usepackage[spanish, mexico, es-noshorthands, english]{babel}
% remove space between margin and lists
\usepackage{enumitem}
\setitemize{label=\usebeamerfont*{itemize item}%
\usebeamercolor[fg]{itemize item}
\usebeamertemplate{itemize item}}
\setlist{leftmargin=*,labelindent=0cm}
\setenumerate[1]{%
label=\protect\usebeamerfont{enumerate item}%
\protect\usebeamercolor[fg]{enumerate item}%
\insertenumlabel.}
\usepackage{tikz}
\usepackage{tikz-cd}
\usepackage{pgfplots}
\usetikzlibrary{babel}
\beamerdefaultoverlayspecification{<+->}
\usefonttheme{professionalfonts}
\usetheme{default}
\usecolortheme{shark}
\useinnertheme[outline,shadow]{chamfered}
\useoutertheme[glossy,nofootline]{wuerzburg}
\date{2015-04-17 7:00}
\title{Series de funciones analíticas}
\begin{document}

\maketitle
\setbeamertemplate{navigation symbols}{}
\setbeamertemplate{items}[circle]
\languagepath{spanish}

\tableofcontents

\section{Repaso de sucesiones y series}
\label{sec-1}

\begin{frame}[label=sec-1-1]{Definiciones}
\begin{definition}[Convergencia]
\begin{itemize}
\item Decimos que la sucesión de números complejos \(z_{n}\) \alert{converge}
      a \(z_{0}\in \mathbb{C}\) si para todo \(\epsilon>0\) existe \(N>0\)
      tal que 
\begin{displaymath}
n\geq N \text{ implica } |z_{n}-z_{0}|<\epsilon.
\end{displaymath}
En tal caso, escribimos \(z_{n}\to z_{0}\).
\item La serie \(\sum_{k=1}^{\infty}a_{k}\) de números complejos \alert{converge}
a \(S\) si la sucesión de sumas parciales
\(s_{n}=\sum_{k=1}^{n}a_{k}\) converge a \(S\). En tal caso
escribimos \(\sum_{k=1}^{\infty}a_{k}=S\).
\end{itemize}
\end{definition}
\end{frame}

\begin{frame}[label=sec-1-2]{}
\begin{theorem}[Criterio de Cauchy]
\begin{itemize}
\item \(z_{n}\) converge si y solo si para todo \(\epsilon>0\) existe
\(N>0\) tal que si \(n,m\geq N\) entonces \(|z_{n}-z_{m}|<\epsilon\).
\item \(\sum_{k=1}^{\infty}a_{k}\) converge si y solo si para todo
\(\epsilon>0\) existe \(N\) tal que si \(n\geq N\), entonces
\begin{displaymath}
\left| \sum_{k=n+1}^{n+p}a_{k} \right|<\epsilon\text{ para \(p=1,2,\ldots\)}
\end{displaymath}
\end{itemize}
\end{theorem}

\begin{corollary}
Si \(\sum_{k=1}^{\infty}a_{k}\) converge, entonces \(a_{n}\to 0\).
\end{corollary}
\end{frame}

\begin{frame}[label=sec-1-3]{}
\begin{definition}[Convergencia absoluta]
Si \(\sum_{k=1}^{\infty}|a_{k}|\) converge, decimos que
\(\sum_{k=1}^{\infty}a_{k}\) \alert{converge absolutamente}.
\end{definition}

\begin{theorem}
Si \(\sum_{k=1}^{\infty}a_{k}\) converge absolutamente, entonces
converge. 
\end{theorem}
\end{frame}

\begin{frame}[label=sec-1-4]{Criterios de convergencia}
\begin{theorem}
\begin{itemize}
\item Si \(|r|<1\), entonces \(\sum_{n=0}^{\infty}r^{n}\) converge a
\(\frac{1}{1-r}\). Si \(|r|\geq 1\), la serie \emph{diverge}.
\item Si \(\sum_{k=1}^{\infty}b_{k}\) converge y \(0\leq a_{k}\leq
      b_{k}\) converge, entonces \(\sum_{k=1}^{\infty}a_{k}\)
      converge. (Resultado dual para divergencia).
\item \(\sum_{n=1}^{\infty}\frac{1}{n^{p}}\) converge si \(p>1\). Si
\(p\leq 1\), la serie \emph{diverge a \(\infty\)}.
\item Si \(\frac{|a_{n+1}|}{|a_{n}|}\to r\) con \(r<1\), entonces
\(\sum_{n=1}^{\infty}a_{n}\) converge absolutamente. Si \(r>1\),
la serie diverge.
\item Si \(|a_{n}|^{\frac{1}{n}}\to r\) y \(r<1\), entonces
\(\sum_{n=1}^{\infty}a_{n}\) converge absolutamente. Si \(r>1\),
la serie diverge.
\end{itemize}
\end{theorem}
\end{frame}


\begin{frame}[label=sec-1-5]{Sucesiones de funciones}
\begin{definition}
\begin{itemize}
\item Sea \(f_{n}\colon D\to \mathbb{C}\) una sucesión de funciones
definidas en \(D\). Si existe \(f\colon D\to \mathbb{C}\) tal
que \(f_{n}(z)\to f(z)\) para todo \(z\in D\), decimos que
\(f_{n}\) \alert{converge puntualmente} a \(f\). En este caso,
escribimos \(f_{n}\to f\).
\item Si para todo \(\epsilon>0\) existe \(N>0\) tal que \(n\geq N\)
implica \(|f_{n}(z)-f(z)|<\epsilon\) para todo \(z\in D\),
decimos que \(f_{n}\) \alert{converge uniformemente} a \(f\).
\item Se dice que la serie \(\sum_{k=1}^{\infty}g_{k}(z)\) converge
puntualmente (uniformemente) si la sucesión de sumas parciales
converge puntualmente (uniformemente).
\end{itemize}
\end{definition}
\end{frame}

\begin{frame}[label=sec-1-6]{}
\begin{theorem}[Criterio de Cauchy]
\begin{itemize}
\item \(f_{n}(z)\) converge uniformemente si y solo si para todo \(\epsilon>0\) existe
\(N>0\) tal que si \(n\geq N\) entonces
\(|f_{n}-f_{n+p}|<\epsilon\) para todo \(p=1,2,\ldots\).
\item \(\sum_{k=1}^{\infty}g_{k}\) converge uniformemente en \(D\) si y solo si para todo
\(\epsilon>0\) existe \(N\) tal que si \(n\geq N\), entonces
\begin{displaymath}
\left| \sum_{k=n+1}^{n+p}g_{k}(z) \right|<\epsilon\text{ para \(z\in D\) y \(p=1,2,\ldots\)}
\end{displaymath}
\end{itemize}
\end{theorem}
\end{frame}

\begin{frame}[label=sec-1-7]{Límite uniforme}
\begin{theorem}
Si cada \(f_{n}\) es continua y \(f_{n}\to f\) uniformemente,
entonces \(f\) es continua.
\end{theorem}
\end{frame}

\begin{frame}[label=sec-1-8]{Criterio de Weierstrass}
\begin{theorem}
Sea \(g_{n}\colon D\to \mathbb{C}\) una sucesión de
funciones. Supongamos que existe una sucesión \(M_{n}\) de reales
\(M_{n}\geq 0\) tal que:

\begin{itemize}
\item \(|g_{n}(z)|\leq M_{n}\) para \(z\in D\),
\item \(\sum_{n=1}^{\infty}M_{n}\) converge
\end{itemize}

Entonces, \(\sum_{n=1}^{\infty}g_{n}\) converge absoluta y
uniformemente en \(D\).
\end{theorem}
\end{frame}


\section{Convergencia de funciones analíticas}
\label{sec-2}

\begin{frame}[label=sec-2-1]{}
\begin{theorem}
\begin{itemize}
\item Sea \(\gamma\colon [a,b]\to D\) una curva en la región \(D\), y
sea \(f_{n}\colon \gamma([a,b])\to \mathbb{C}\) una sucesión de
funciones continuas, tal que \(f_{n}\to f\) uniformemente, donde
\(f\colon \gamma([a,b])\to \mathbb{C}\). Entonces:
\begin{displaymath}
\int_{\gamma}f_{n}\,dz\to \int_{\gamma}f\,dz.
\end{displaymath}
\item Si \(\sum_{n=1}^{\infty}g_{n}\) converge uniformemente en
\(\gamma([a,b])\), entonces:
\begin{displaymath}
\int_{\gamma}\left(\sum_{n=1}^{\infty}g_{n}(z)\right)dz=\sum_{n=1}^{\infty}\left(\int_{\gamma} g_{n}(z)\,dz\right).
\end{displaymath}
\end{itemize}
\end{theorem}
\end{frame}

\begin{frame}[label=sec-2-2]{}
\begin{theorem}
\begin{itemize}
\item Sean \(D\) abierto y \(f_{n}\colon D\to \mathbb{C}\) una
sucesión de funciones analíticas. Si \(f_{n}\to f\)
uniformemente en cada disco cerrado contenido en \(D\), entonces
\(f\) es analítica. Además \(f_{n}'\to f'\) puntualmente en
\(D\) y uniformemente en cada disco cerrado contenido en \(D\).

\item Sea \(g_{k}\colon D\to \mathbb{C}\) una sucesión de funciones
analíticas tal que \(g(z)=\sum_{k=1}^{\infty}g_{k}(z)\) con
convergencia uniforme en cada disco cerrado contenido en
\(D\). Entonces \(g\colon D\to \mathbb{C}\) es analítica, y
\(g'(z)=\sum_{k=1}^{\infty}g_{k}'(z)\) puntualmente en \(D\) y
uniformemente en cada disco cerrado contenido en \(D\).
\end{itemize}
\end{theorem}
\end{frame}
% Emacs 24.3.1 (Org mode N/A)
\end{document}