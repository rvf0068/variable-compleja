% Created 2015-03-09 lun 08:09
\documentclass[spanish,presentation]{beamer}
\usepackage[utf8]{inputenc}
\usepackage[T1]{fontenc}
\usepackage{fixltx2e}
\usepackage{graphicx}
\usepackage{longtable}
\usepackage{float}
\usepackage{wrapfig}
\usepackage{rotating}
\usepackage[normalem]{ulem}
\usepackage{amsmath}
\usepackage{textcomp}
\usepackage{marvosym}
\usepackage{wasysym}
\usepackage{amssymb}
\usepackage{capt-of}
\tolerance=1000
\usepackage{listings}
\usepackage[mono=false]{libertine}
\usepackage[scaled=0.7]{luximono}
\usepackage{lxfonts}
\usepackage[spanish, mexico, es-noshorthands, english]{babel}
% remove space between margin and lists
\usepackage{enumitem}
\setitemize{label=\usebeamerfont*{itemize item}%
\usebeamercolor[fg]{itemize item}
\usebeamertemplate{itemize item}}
\setlist{leftmargin=*,labelindent=0cm}
\setenumerate[1]{%
label=\protect\usebeamerfont{enumerate item}%
\protect\usebeamercolor[fg]{enumerate item}%
\insertenumlabel.}
\usepackage{tikz}
\usepackage{tikz-cd}
\usepackage{pgfplots}
\usetikzlibrary{babel}
\beamerdefaultoverlayspecification{<+->}
\usefonttheme{professionalfonts}
\usetheme{default}
\usecolortheme{shark}
\useinnertheme[outline,shadow]{chamfered}
\useoutertheme[glossy,nofootline]{wuerzburg}
\date{2015-02-16 9:00}
\title{Teoremas sobre integrales}
\begin{document}

\maketitle
\setbeamertemplate{navigation symbols}{}
\setbeamertemplate{items}[circle]
\languagepath{spanish}

\tableofcontents

\section{Teorema fundamental}
\label{sec-1}

\begin{frame}[label=sec-1-1]{}
\begin{theorem}
Sea \(f\colon U\to \mathbb{C}\) continua, y sea \(F\colon U\to
    \mathbb{C}\) tal que \(F'=f\). Si \(\gamma\colon[a,b]\to U\) es
suave a trozos, entonces:
\begin{displaymath}
\int_{\gamma}f(z)\,dz=[F(z)]_{\gamma(a)}^{\gamma(b)}=F(\gamma(b))-F(\gamma(a)).
\end{displaymath}
\end{theorem}

\begin{block}{}
En particular, si \(\gamma\) es una curva cerrada, se obtiene que
\(\int_{\gamma}f(z)\,dz=0\). 
\end{block}
\end{frame}

\section{Longitud}
\label{sec-2}

\begin{frame}[label=sec-2-1]{}
\begin{block}{Longitud}
Sea \(\gamma\colon[a,b]\to U\) una curva suave. Su \alert{longitud} se
define como:
\begin{displaymath}
\int_{\gamma}|dz|=\int_{a}^{b}|\gamma'(t)|\,dt
\end{displaymath}
\end{block}

\begin{block}{}
Sea \(f\colon U\to \mathbb{C}\) continua. La \alert{integral de \(f\)
sobre \(\gamma\) respecto a longitud de arco} se define como:
\begin{displaymath}
\int_{\gamma} f(z)\,|dz|=\int_{a}^{b} f(\gamma(t))|\gamma'(t)|\,dt
\end{displaymath}
\end{block}
\end{frame}


\section{Teorema de estimación}
\label{sec-3}

\begin{frame}[label=sec-3-1]{}
\begin{theorem}
Sean \(f\colon U\to \mathbb{C}\) continua y \(\gamma\) una curva
suave a trozos en en \(U\). Entonces:
\begin{displaymath}
\left|\int_{\gamma}f(z)\,dz\right|\leq \int_{\gamma}|f(z)|\,|dz|
\end{displaymath}
\end{theorem}
\end{frame}
% Emacs 24.3.1 (Org mode N/A)
\end{document}