% Created 2015-02-13 vie 10:13
\documentclass[spanish,presentation]{beamer}
\usepackage[utf8]{inputenc}
\usepackage[T1]{fontenc}
\usepackage{fixltx2e}
\usepackage{graphicx}
\usepackage{longtable}
\usepackage{float}
\usepackage{wrapfig}
\usepackage{rotating}
\usepackage[normalem]{ulem}
\usepackage{amsmath}
\usepackage{textcomp}
\usepackage{marvosym}
\usepackage{wasysym}
\usepackage{amssymb}
\usepackage{capt-of}
\tolerance=1000
\usepackage{listings}
\usepackage[mono=false]{libertine}
\usepackage[scaled=0.7]{luximono}
\usepackage{lxfonts}
\usepackage[spanish, mexico, es-noshorthands, english]{babel}
% remove space between margin and lists
\usepackage{enumitem}
\setitemize{label=\usebeamerfont*{itemize item}%
\usebeamercolor[fg]{itemize item}
\usebeamertemplate{itemize item}}
\setlist{leftmargin=*,labelindent=0cm}
\setenumerate[1]{%
label=\protect\usebeamerfont{enumerate item}%
\protect\usebeamercolor[fg]{enumerate item}%
\insertenumlabel.}
\usepackage{tikz}
\usepackage{tikz-cd}
\usepackage{pgfplots}
\usetikzlibrary{babel}
\beamerdefaultoverlayspecification{<+->}
\usefonttheme{professionalfonts}
\usetheme{default}
\usecolortheme{shark}
\useinnertheme[outline,shadow]{chamfered}
\useoutertheme[glossy,nofootline]{wuerzburg}
\date{2015-02-09 9:00}
\title{La función exponencial}
\begin{document}

\maketitle
\setbeamertemplate{navigation symbols}{}
\setbeamertemplate{items}[circle]
\languagepath{spanish}

\tableofcontents

\section{Definición}
\label{sec-1}

\begin{frame}[label=sec-1-1]{}
\begin{block}{}
Observemos que las funciones \(u=e^{x}\cos y\) y \(v=e^{x}\sin y\)
satisfacen las ecuaciones de Cauchy-Riemann en el plano complejo
\(\mathbb{C}\). 
\end{block}

\begin{definition}
La función de variable compleja
\begin{displaymath}
e^{z}=e^{x+iy}=e^{x}\cos y+ie^{x}\sin y
\end{displaymath}
se llama \alert{función exponencial compleja}.
\end{definition}
\end{frame}

\begin{frame}[label=sec-1-2]{Propiedades}
\begin{itemize}
\item \((e^{z})'=e^{z}\).
\item \(e^{a+b}=e^{a}e^{b}\) para todos \(a,b\in \mathbb{C}\).
\item \(e^{z}\ne 0\) para todo \(z\in \mathbb{C}\).
\item \(f(z)=e^{z}\) es una función \alert{entera}, es decir, es analítica en
el plano complejo \(\mathbb{C}\).
\end{itemize}
\end{frame}

\section{Las funciones trigonométricas}
\label{sec-2}

\begin{frame}[label=sec-2-1]{}
\begin{definition}[Seno y coseno]
Las funciones trigonométricas de variable compleja se definen
como:
\begin{displaymath}
\cos z=\frac{e^{iz}+e^{-iz}}{2},\qquad \sin z=\frac{e^{iz}-e^{-iz}}{2i}
\end{displaymath}
\end{definition}

\begin{block}{Propiedades}
\begin{itemize}
\item \(e^{iz}=\cos z+i\sin z\) (fórmula de Euler),
\item \(\cos^{2} z+\sin^{2}z=1\),
\item \((\cos z)'=-\sin z\), \((\sin z)'=\cos z\),
\item \(\cos(a+b)=\cos a\cos b-\sin a\sin b\), \(\sin(a+b)=\cos a\sin
      b+\sin a\cos b\).
\end{itemize}
\end{block}
\end{frame}

\section{Logaritmo}
\label{sec-3}

\begin{frame}[label=sec-3-1]{Definición}
\begin{definition}[Logaritmo]
Dado \(w\in \mathbb{C}\), se define \(z=\log w\) como una solución
de la ecuación \(e^{z}=w\).
\end{definition}
\end{frame}

\begin{frame}[label=sec-3-2]{}
\begin{block}{Observaciones}
\begin{itemize}
\item Como \(e^{z}\ne 0\) para todo \(z\in \mathbb{C}\), se obtiene
que \(0\) no tiene logaritmo.
\item Sea \(w\ne 0\). Entonces \(e^{x+iy}=w\) es equivalente a:
\begin{displaymath}
e^{x}=|w|,\qquad e^{iy}=\frac{w}{|w|}.
\end{displaymath}
\item De lo anterior se obtiene:
\begin{displaymath}
\log w = \log |w| +i\arg w,
\end{displaymath}
donde \(\arg w\) representa el conjunto de argumentos de
\(w\). En particular, \(\log w\) tiene una infinidad de valores,
que difieren por un múltiplo de \(2\pi i\).
\item Sin embargo, si \(a\in\mathbb{R}\), \(a>0\), consideraremos
el valor usual (real) de \(\log a\).
\end{itemize}
\end{block}
\end{frame}

\begin{frame}[label=sec-3-3]{Ramas}
\begin{definition}
Si \(f\colon U\to \mathbb{C}\) es analítica y \(D\subseteq f(U)\)
es un dominio, una \alert{rama de \(f^{-1}\) en \(D\)} es una función
continua \(g\colon D\to U\) tal que \(f(g(z))=z\) para todo \(z\in
    D\).
\end{definition}

\begin{definition}[Argumento principal]
Dado \(z\in \mathbb{C}-\{0\}\), definimos
\(\mathrm{Arg}(z)\) como el argumento de \(z\) que está en el
intervalo \((-\pi,\pi]\).
\end{definition}

\begin{block}{Observación}
La función \(\mathrm{Arg}\colon \mathbb{C}-\{z\in \mathbb{C}\mid
    z\leq 0\}\to \mathbb{R}\) es continua.
\end{block}
\end{frame}

\begin{frame}[label=sec-3-4]{Raíz cuadrada}
\begin{example}
La función
\begin{displaymath}
z\mapsto \sqrt{|z|}(\cos \frac{\mathrm{Arg}(z)}{2}+i\sin\frac{\mathrm{Arg}(z)}{2}),
\end{displaymath}
es una rama de la inversa de \(f(z)=z^{2}\), definida en
\(D=\mathbb{C}-\{z\in \mathbb{C}\mid z\leq 0\}\).
\end{example}

\begin{example}
La función 
\begin{displaymath}
z\mapsto \log |z|+i\mathrm{Arg}(z)
\end{displaymath}
es una rama de la inversa de \(f(z)=e^{z}\), definida en
\(D=\mathbb{C}-\{z\in \mathbb{C}\mid z\leq 0\}\).
\end{example}
\end{frame}

\begin{frame}[label=sec-3-5]{Derivabilidad de ramas}
\begin{theorem}
Sea \(f\colon U\to \mathbb{C}\) analítica, y sea \(g\colon D\to
    U\) una rama de \(f^{-1}\). Sean \(z_{0}\in D\) y
\(w_{0}=g(z_{0})\in U\). Si \(f'(w_{0})\ne 0\), entonces \(g\) es
derivable en \(z_{0}\) y \(g'(z_{0})=\frac{1}{f'(w_{0})}\).

Por lo tanto, si \(f'\) no tiene ceros en \(g(D)\), entonces \(g\)
es analítica en \(D\), y \(g'(z)=\frac{1}{f'(g(z))}\).
\end{theorem}
\end{frame}
% Emacs 24.3.1 (Org mode N/A)
\end{document}