% Created 2015-02-09 lun 08:38
\documentclass[spanish,presentation]{beamer}
\usepackage[utf8]{inputenc}
\usepackage[T1]{fontenc}
\usepackage{fixltx2e}
\usepackage{graphicx}
\usepackage{longtable}
\usepackage{float}
\usepackage{wrapfig}
\usepackage{rotating}
\usepackage[normalem]{ulem}
\usepackage{amsmath}
\usepackage{textcomp}
\usepackage{marvosym}
\usepackage{wasysym}
\usepackage{amssymb}
\usepackage{capt-of}
\tolerance=1000
\usepackage{listings}
\usepackage[mono=false]{libertine}
\usepackage[scaled=0.7]{luximono}
\usepackage{lxfonts}
\usepackage[spanish, mexico, es-noshorthands, english]{babel}
% remove space between margin and lists
\usepackage{enumitem}
\setitemize{label=\usebeamerfont*{itemize item}%
\usebeamercolor[fg]{itemize item}
\usebeamertemplate{itemize item}}
\setlist{leftmargin=*,labelindent=0cm}
\setenumerate[1]{%
label=\protect\usebeamerfont{enumerate item}%
\protect\usebeamercolor[fg]{enumerate item}%
\insertenumlabel.}
\usepackage{tikz}
\usepackage{tikz-cd}
\usepackage{pgfplots}
\usetikzlibrary{babel}
\beamerdefaultoverlayspecification{<+->}
\usefonttheme{professionalfonts}
\usetheme{default}
\usecolortheme{shark}
\useinnertheme[outline,shadow]{chamfered}
\useoutertheme[glossy,nofootline]{wuerzburg}
\date{2015-02-09 9:00}
\title{La función exponencial}
\begin{document}

\maketitle
\setbeamertemplate{navigation symbols}{}
\setbeamertemplate{items}[circle]
\languagepath{spanish}

\tableofcontents

\section{Definición}
\label{sec-1}

\begin{frame}[label=sec-1-1]{}
\begin{block}{}
Observemos que las funciones \(u=e^{x}\cos y\) y \(v=e^{x}\sin y\)
satisfacen las ecuaciones de Cauchy-Riemann en el plano complejo
\(\mathbb{C}\). 
\end{block}

\begin{definition}
La función de variable compleja
\begin{displaymath}
f(z)=f(x+iy)=e^{x}\cos y+ie^{x}\sin y
\end{displaymath}
se llama \alert{función exponencial compleja}.
\end{definition}
\end{frame}

\begin{frame}[label=sec-1-2]{Propiedades}
\begin{itemize}
\item \((e^{z})'=e^{z}\).
\item \(e^{a+b}=e^{a}e^{b}\) para todos \(a,b\in \mathbb{C}\).
\item \(e^{z}\ne 0\) para todo \(z\in \mathbb{C}\).
\item \(f(z)=e^{z}\) es una función \alert{entera}, es decir, es analítica en
\(\mathbb{C}\).
\end{itemize}
\end{frame}

\section{Las funciones trigonométricas}
\label{sec-2}

\begin{frame}[label=sec-2-1]{}
\begin{definition}
Las funciones trigonométricas de variable compleja se definen
como:
\begin{displaymath}
\cos z=\frac{e^{iz}+e^{-iz}}{2},\qquad \sin z=\frac{e^{iz}-e^{-iz}}{2i}
\end{displaymath}
\end{definition}

\begin{block}{Propiedades}
\begin{itemize}
\item \(e^{iz}=\cos z+i\sin z\) (fórmula de Euler),
\item \(\cos^{2} z+\sin^{2}z=1\),
\item \((\cos z)'=-\sin z\), \((\sin z)'=\cos z\),
\item \(\cos(a+b)=\cos a\cos b-\sin a\sin b\), \(\sin(a+b)=\cos a\sin
      b+\sin a\cos b\).
\end{itemize}
\end{block}
\end{frame}
% Emacs 24.3.1 (Org mode N/A)
\end{document}