% Created 2015-05-07 jue 16:19
\documentclass[spanish,presentation]{beamer}
\usepackage[utf8]{inputenc}
\usepackage[T1]{fontenc}
\usepackage{fixltx2e}
\usepackage{graphicx}
\usepackage{longtable}
\usepackage{float}
\usepackage{wrapfig}
\usepackage{rotating}
\usepackage[normalem]{ulem}
\usepackage{amsmath}
\usepackage{textcomp}
\usepackage{marvosym}
\usepackage{wasysym}
\usepackage{amssymb}
\usepackage{capt-of}
\tolerance=1000
\usepackage{listings}
\usepackage[mono=false]{libertine}
\usepackage[scaled=0.7]{luximono}
\usepackage{lxfonts}
\usepackage[spanish, mexico, es-noshorthands, english]{babel}
% remove space between margin and lists
\usepackage{enumitem}
\setitemize{label=\usebeamerfont*{itemize item}%
\usebeamercolor[fg]{itemize item}
\usebeamertemplate{itemize item}}
\setlist{leftmargin=*,labelindent=0cm}
\setenumerate[1]{%
label=\protect\usebeamerfont{enumerate item}%
\protect\usebeamercolor[fg]{enumerate item}%
\insertenumlabel.}
\usepackage{tikz}
\usepackage{tikz-cd}
\usepackage{pgfplots}
\usetikzlibrary{babel}
\usetheme{default}
\usecolortheme{shark}
\useinnertheme[outline,shadow]{chamfered}
\useoutertheme[glossy,nofootline]{wuerzburg}
\date{2015-05-04 9:00}
\title{Series de Laurent}
\beamerdefaultoverlayspecification{<+->}
\usefonttheme{professionalfonts}
\hypersetup{
 pdfauthor={},
 pdftitle={Series de Laurent},
 pdfkeywords={},
 pdfsubject={},
 pdfcreator={Emacs 24.3.1 (Org mode N/A)}, 
 pdflang={English}}
\begin{document}

\maketitle
\setbeamertemplate{navigation symbols}{}
\setbeamertemplate{items}[circle]
\languagepath{spanish}

\tableofcontents

\section{Definición}
\label{orgheadline1}

\begin{frame}[label=sec-1-1]{}
\begin{theorem}[Expansión de Laurent]
Sean \(r_{1},r_{2}\) tales que \(0\leq r_{1}<r_{2}\), \(z_{0}\in
    \mathbb{C}\). Sea \(A=\{z\in \mathbb{C}\mid
    r_{1}<|z-z_{0}|<r_{2}\}\). Sea \(f\) analítica en \(A\). Entonces
existen \(a_{n},b_{n}\in \mathbb{C}\) tales que:
\begin{displaymath}
f(z)=\sum_{n=0}^{\infty}a_{n}(z-z_{0})^{n}+\sum_{n=1}^{\infty}\frac{b_{n}}{(z-z_{0})^{n}},
\end{displaymath}
donde ambas series del lado derecho convergen absolutamente en
\(A\) y uniformemente en conjuntos de la forma
\(B_{\rho_{1},\rho_{2}}=\{z\in \mathbb{C}\mid
    \rho_{1}\leq|z-z_{0}|\leq\rho_{2}\}\), donde
\(r_{1}<\rho_{1}<\rho_{2}<r_{2}\).
\end{theorem}
\end{frame}

\begin{frame}[label=sec-1-2]{}
\begin{theorem}[Continuación]
Si \(\gamma\) es un círculo centrado en \(z_{0}\) de radio \(r\),
con \(r_{1}<r<r_{2}\), los coeficientes \(a_{n},b_{n}\) están
dados por:
\begin{displaymath}
a_{n}=\frac{1}{2\pi i}\int_{\gamma}\frac{f(\xi)}{(\xi-z_{0})^{n+1}}d\xi,
\end{displaymath}
\begin{displaymath}
b_{n}=\frac{1}{2\pi i}\int_{\gamma}f(\xi)(\xi-z_{0})^{n-1}\,d\xi
\end{displaymath}
Además, la \emph{expansión de Laurent} de \(f\) en \(A\) es única.
\end{theorem}
\end{frame}

\section{}
\label{orgheadline1}

\begin{frame}[label=sec-2-1]{}
\begin{block}{Observación}
La expansión de Laurent con \(r_{1}=0\) se utiliza para el estudio
de singularidades.
\end{block}
\end{frame}

\section{}
\label{orgheadline1}
\begin{frame}[label=sec-3-1]{}
\begin{block}{Residuo}
Si \(f\) tiene una singularidad aislada en \(z_{0}\), el valor del
coeficiente \(b_{1}\) en la serie de Laurent definida en un disco
perforado alrededor de \(z_{0}\) se llama el \emph{residuo}
de \(f\) en \(z_{0}\).
\end{block}
\end{frame}
\end{document}