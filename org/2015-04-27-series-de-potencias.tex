% Created 2015-04-27 Mon 11:25
\documentclass[spanish,presentation]{beamer}
\usepackage[utf8]{inputenc}
\usepackage[T1]{fontenc}
\usepackage{fixltx2e}
\usepackage{graphicx}
\usepackage{longtable}
\usepackage{float}
\usepackage{wrapfig}
\usepackage{rotating}
\usepackage[normalem]{ulem}
\usepackage{amsmath}
\usepackage{textcomp}
\usepackage{marvosym}
\usepackage{wasysym}
\usepackage{amssymb}
\usepackage{capt-of}
\tolerance=1000
\usepackage{listings}
\usepackage[mono=false]{libertine}
\usepackage[scaled=0.7]{luximono}
\usepackage{lxfonts}
\usepackage[spanish, mexico, es-noshorthands, english]{babel}
% remove space between margin and lists
\usepackage{enumitem}
\setitemize{label=\usebeamerfont*{itemize item}%
\usebeamercolor[fg]{itemize item}
\usebeamertemplate{itemize item}}
\setlist{leftmargin=*,labelindent=0cm}
\setenumerate[1]{%
label=\protect\usebeamerfont{enumerate item}%
\protect\usebeamercolor[fg]{enumerate item}%
\insertenumlabel.}
\usepackage{tikz}
\usepackage{tikz-cd}
\usepackage{pgfplots}
\usetikzlibrary{babel}
\usetheme{default}
\usecolortheme{shark}
\useinnertheme[outline,shadow]{chamfered}
\useoutertheme[glossy,nofootline]{wuerzburg}
\date{2015-04-20 9:00}
\title{Series de potencias}
\beamerdefaultoverlayspecification{<+->}
\usefonttheme{professionalfonts}
\hypersetup{
 pdfauthor={},
 pdftitle={Series de potencias},
 pdfkeywords={},
 pdfsubject={},
 pdfcreator={Emacs 24.3.1 (Org mode 8.3beta)}, 
 pdflang={English}}
\begin{document}

\maketitle
\setbeamertemplate{navigation symbols}{}
\setbeamertemplate{items}[circle]
\languagepath{spanish}

\tableofcontents

\section{Definición}
\label{orgheadline1}

\begin{frame}[label=sec-1-1]{}
\begin{definition}[Serie de potencias]
Una \alert{serie de potencias} es una serie de la forma:
\begin{displaymath}
\sum_{n=0}^{\infty}a_{n}(z-z_{0})^{n},
\end{displaymath}
donde los \(a_{n}\) y \(z_{0}\) son números complejos.
\end{definition}
\end{frame}


\section{Teoremas}
\label{orgheadline1}

\begin{frame}[label=sec-2-1]{}
\begin{lemma}[Abel-Weierstrass]
Sean \(r_{0}\geq0\) y \(M\) una constante tal que
\(|a_{n}|r_{0}^{n}\leq M\) para todo \(n\geq0\). Entonces, para
cada \(r<r_{0}\) la serie
\(\sum_{n=0}^{\infty}a_{n}(z-z_{0})^{n}\) converge uniformemente y
absolutamente en \(D_{r}=\{z\in \mathbb{C}\mid |z-z_{0}|\leq r\}\).
\end{lemma}

\begin{theorem}[Convergencia de series de potencias]
Dada una serie de potencias
\(\sum_{n=0}^{\infty}a_{n}(z-z_{0})^{n}\), existe un único número
\(R\geq0\), (posiblemente \(+\infty\)), llamado \alert{radio de
convergencia}, tal que si \(|z-z_{0}|<R\) la serie converge, y si
\(|z-z_{0}|>R\), la serie diverge. Además, la convergencia es
uniforme en cada disco cerrado contenido en \(D=\{z\in
    \mathbb{C}\mid |z-z_{0}|< R\}\). 
\end{theorem}
\end{frame}

\begin{frame}[label=sec-2-2]{Corolario}
\begin{corollary}[Serie es analítica]
Una serie de potencias \(\sum_{n=0}^{\infty}a_{n}(z-z_{0})^{n}\)
es una función analítica en su \emph{círculo de convergencia} \(D=\{z\in
    \mathbb{C}\mid |z-z_{0}|< R\}\). 
\end{corollary}
\end{frame}

\begin{frame}[label=sec-2-3]{}
\begin{theorem}[Derivación de series de potencias]
Sea
\begin{displaymath}
f(z)=\sum_{n=0}^{\infty}a_{n}(z-z_{0})^{n},
\end{displaymath}
una función analítica dada por una serie de potencias en su
círculo de convergencia. Entonces:
\begin{itemize}
\item \(f'(z)=\sum_{n=1}^{\infty}na_{n}(z-z_{0})^{n-1}\) y la
serie para \(f'(z)\) tiene el mismo radio de convergencia que la
serie para \(f(z)\).
\item Además, \(a_{n}=\frac{f^{(n)}(z_{0})}{n!}\).
\end{itemize}
\end{theorem}
\end{frame}

\begin{frame}[label=sec-2-4]{}
\begin{corollary}[Unicidad]
Si
\begin{displaymath}
\sum_{n=0}^{\infty}a_{n}(z-z_{0})^{n}=f(z)=\sum_{n=0}^{\infty}b_{n}(z-z_{0})^{n}
\end{displaymath}
para todo \(z\) en \(D(z_{0},r)\) con \(r>0\), entonces
\(a_{n}=b_{n}\) para toda \(n\).
\end{corollary}
\end{frame}

\begin{frame}[label=sec-2-5]{}
\begin{theorem}[Cálculo del radio de convergencia]
Dada la serie de potencias
\(\sum_{n=0}^{\infty}a_{n}(z-z_{0})^{n}\):

\begin{itemize}
\item si
\begin{displaymath}
\lim_{n\to\infty}\frac{|a_{n}|}{|a_{n+1}|}
\end{displaymath}
existe, es igual a \(R\), el radio de convergencia de la serie.
\item Si \(\rho=\lim_{n\to\infty} \sqrt[n]{|a_{n}|}\) existe, entonces
\(R=\frac{1}{\rho}\) es el radio de convergencia. (\(R=\infty\)
si \(\rho=0\)).
\end{itemize}
\end{theorem}
\end{frame}

\section{Teorema de Taylor}
\label{orgheadline1}

\begin{frame}[label=sec-3-1]{}
\begin{theorem}[Taylor]
Sea \(f\colon A\to \mathbb{C}\) analítica con \(A\subseteq
    \mathbb{C}\) abierto. Sean \(z_{0}\in A\) y \(r>0\) tal que
\(D_{r}=\{z\mid |z-z_{0}|<r\}\subseteq A\). Entonces, para cada
\(z\in D_{r}\), la serie:
\begin{displaymath}
\sum_{n=0}^{\infty} \frac{f^{n}(z_{0})}{n!}(z-z_{0})^{n}
\end{displaymath}
converge en \(A\), y además: \(\sum_{n=0}^{\infty}
    \frac{f^{n}(z_{0})}{n!}(z-z_{0})^{n}=f(z)\).
\end{theorem}
\end{frame}

\begin{frame}[label=sec-3-2]{}
\begin{corollary}[Desarrollo en series de potencias]
Sea \(A\subseteq \mathbb{C}\) abierto y sea \(f\colon A\to
    \mathbb{C}\). Entonces \(f\) es analítica en \(A\) si y solo si
para cada \(z_{0}\in A\) existe \(r>0\) tal que
\(D(z_{0},r)\subseteq A\) y \(f\) es igual a una serie de
potencias convergente en \(D(z_{0},r)\).
\end{corollary}

\begin{block}{}
\begin{itemize}
\item \(e^{z}=\sum_{n=0}^{\infty}\frac{z^{n}}{n!}\),
\item \(\sin z=\sum_{n=1}^{\infty}(-1)^{n+1}\frac{z^{2n-1}}{(2n-1)!}\).
\end{itemize}
\end{block}
\end{frame}
\end{document}